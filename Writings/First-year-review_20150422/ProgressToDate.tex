\chapter{Progress To Date}
\label{Chp: Progress To Date}

As a beginning of this project, our first step is to demonstrate that information leakage similar to those described in \cite{Web1} can be found when the underlying protocol is switched DTLS. The reason is that DTLS is more suitable to sensor networks comparing to TLS due to is relatively lightweight-ness. 

The basic idea is to build some toy applications which model typical sensor network traffic generated through DTLS. The information leakage of toy applications are intentionally crafted to emphasise their existence.

Even though both OpenSSL and GnuSSL have DTLS implemented with general features, we setted our experimental the less featured tinyDTLS\cite{tinyDTLS} due to its lightweight-ness, which is more suitable to sensor networks. However, the drawback is that only one cipher-suite, TLS\_ECDHE\_ECDSA\_WITH\_AES\_128\_CCM\_8\cite{rfc7251}, is available for the current version of tinyDTLS. This implies that there should be no padding scheme adopted and hence the length of plaintext and ciphertext are expected to have a linear relationship. Our experiments supported this conjecture such that:
\begin{equation} \label{Eq: Plaintext length}
l_D = 17 + l
\end{equation}
where $l$ is the length of plaintext and $l_D$ is the value in DTLS length field. According to the specifications the additional bytes is supposed to be purely a resulted of the appending MAC even though $17$ bytes is a value unlikely to be. This problem is still under investigating.

All experiments are done with only two processes, a server and a client referred as SERVER and CLIENT, on a same linux host connected through local-link. The protocol suite we adopted is [IPv4 or IPv6] + UDP + DTLS. The modelled adversary is simply a passive eavesdropper.

So far we have built up two toy applications, \textbf{Odd or Even} and \textbf{Leaky Coffee}, which will be explained in the next sessions alongside with corresponding traffic analysis.

\section{Odd or Even} \label{Sec: Odd or Even}
\textbf{Odd or Even} is a simple toy application designed  to demonstrate the fundamental idea of encrypted traffic analysis.

\subsection{Application Description}

\begin{figure}[H] 
\centering
\resizebox{8cm}{!}
{% Graphic for TeX using PGF
% Title: /home/yy12135/MyGit/tinyDTLS-Traffic-Analysis/Writings/First-year-review_20150422/Pics/OddOrEven.dia
% Creator: Dia v0.97.2
% CreationDate: Mon Mar  2 17:12:59 2015
% For: yy12135
% \usepackage{tikz}
% The following commands are not supported in PSTricks at present
% We define them conditionally, so when they are implemented,
% this pgf file will use them.
\ifx\du\undefined
  \newlength{\du}
\fi
\setlength{\du}{15\unitlength}
\begin{tikzpicture}
\pgftransformxscale{1.000000}
\pgftransformyscale{-1.000000}
\definecolor{dialinecolor}{rgb}{0.000000, 0.000000, 0.000000}
\pgfsetstrokecolor{dialinecolor}
\definecolor{dialinecolor}{rgb}{1.000000, 1.000000, 1.000000}
\pgfsetfillcolor{dialinecolor}
% setfont left to latex
\definecolor{dialinecolor}{rgb}{0.000000, 0.000000, 0.000000}
\pgfsetstrokecolor{dialinecolor}
\node[anchor=west] at (11.000000\du,9.000000\du){};
\definecolor{dialinecolor}{rgb}{1.000000, 1.000000, 1.000000}
\pgfsetfillcolor{dialinecolor}
\pgfpathellipse{\pgfpoint{31.003315\du}{10.006491\du}}{\pgfpoint{1.831366\du}{0\du}}{\pgfpoint{0\du}{1.837164\du}}
\pgfusepath{fill}
\pgfsetlinewidth{0.100000\du}
\pgfsetdash{}{0pt}
\pgfsetdash{}{0pt}
\pgfsetmiterjoin
\definecolor{dialinecolor}{rgb}{0.000000, 0.000000, 0.000000}
\pgfsetstrokecolor{dialinecolor}
\pgfpathellipse{\pgfpoint{31.003315\du}{10.006491\du}}{\pgfpoint{1.831366\du}{0\du}}{\pgfpoint{0\du}{1.837164\du}}
\pgfusepath{stroke}
% setfont left to latex
\definecolor{dialinecolor}{rgb}{0.000000, 0.000000, 0.000000}
\pgfsetstrokecolor{dialinecolor}
\node at (31.003315\du,10.201491\du){SERVER};
\pgfsetlinewidth{0.100000\du}
\pgfsetdash{}{0pt}
\pgfsetdash{}{0pt}
\pgfsetbuttcap
{
\definecolor{dialinecolor}{rgb}{0.000000, 0.000000, 0.000000}
\pgfsetfillcolor{dialinecolor}
% was here!!!
\pgfsetarrowsend{stealth}
\definecolor{dialinecolor}{rgb}{0.000000, 0.000000, 0.000000}
\pgfsetstrokecolor{dialinecolor}
\draw (31.002868\du,11.893189\du)--(31.000000\du,24.000000\du);
}
\pgfsetlinewidth{0.100000\du}
\pgfsetdash{}{0pt}
\pgfsetdash{}{0pt}
\pgfsetbuttcap
{
\definecolor{dialinecolor}{rgb}{0.000000, 0.000000, 0.000000}
\pgfsetfillcolor{dialinecolor}
% was here!!!
\pgfsetarrowsend{to}
\definecolor{dialinecolor}{rgb}{0.000000, 0.000000, 0.000000}
\pgfsetstrokecolor{dialinecolor}
\draw (15.036382\du,13.528274\du)--(31.002048\du,15.352278\du);
}
\pgfsetlinewidth{0.100000\du}
\pgfsetdash{}{0pt}
\pgfsetdash{}{0pt}
\pgfsetbuttcap
{
\definecolor{dialinecolor}{rgb}{0.000000, 0.000000, 0.000000}
\pgfsetfillcolor{dialinecolor}
% was here!!!
\pgfsetarrowsend{to}
\definecolor{dialinecolor}{rgb}{0.000000, 0.000000, 0.000000}
\pgfsetstrokecolor{dialinecolor}
\draw (31.001229\du,18.811367\du)--(15.012127\du,20.509425\du);
}
\definecolor{dialinecolor}{rgb}{1.000000, 1.000000, 1.000000}
\pgfsetfillcolor{dialinecolor}
\pgfpathellipse{\pgfpoint{15.048531\du}{10.031365\du}}{\pgfpoint{1.729717\du}{0\du}}{\pgfpoint{0\du}{1.701399\du}}
\pgfusepath{fill}
\pgfsetlinewidth{0.100000\du}
\pgfsetdash{}{0pt}
\pgfsetdash{}{0pt}
\pgfsetmiterjoin
\definecolor{dialinecolor}{rgb}{0.000000, 0.000000, 0.000000}
\pgfsetstrokecolor{dialinecolor}
\pgfpathellipse{\pgfpoint{15.048531\du}{10.031365\du}}{\pgfpoint{1.729717\du}{0\du}}{\pgfpoint{0\du}{1.701399\du}}
\pgfusepath{stroke}
% setfont left to latex
\definecolor{dialinecolor}{rgb}{0.000000, 0.000000, 0.000000}
\pgfsetstrokecolor{dialinecolor}
\node at (15.048531\du,10.226365\du){CLIENT};
\pgfsetlinewidth{0.100000\du}
\pgfsetdash{}{0pt}
\pgfsetdash{}{0pt}
\pgfsetbuttcap
{
\definecolor{dialinecolor}{rgb}{0.000000, 0.000000, 0.000000}
\pgfsetfillcolor{dialinecolor}
% was here!!!
\pgfsetarrowsend{stealth}
\definecolor{dialinecolor}{rgb}{0.000000, 0.000000, 0.000000}
\pgfsetstrokecolor{dialinecolor}
\draw (15.042445\du,11.782986\du)--(15.000000\du,24.000000\du);
}
\pgfsetlinewidth{0.100000\du}
\pgfsetdash{}{0pt}
\pgfsetdash{}{0pt}
\pgfsetbuttcap
\pgfsetmiterjoin
\pgfsetlinewidth{0.100000\du}
\pgfsetbuttcap
\pgfsetmiterjoin
\pgfsetdash{}{0pt}
\definecolor{dialinecolor}{rgb}{1.000000, 1.000000, 1.000000}
\pgfsetfillcolor{dialinecolor}
\pgfpathmoveto{\pgfpoint{20.833333\du}{13.000000\du}}
\pgfpathlineto{\pgfpoint{24.166667\du}{13.000000\du}}
\pgfpathcurveto{\pgfpoint{24.626904\du}{13.000000\du}}{\pgfpoint{25.000000\du}{13.447715\du}}{\pgfpoint{25.000000\du}{14.000000\du}}
\pgfpathcurveto{\pgfpoint{25.000000\du}{14.552285\du}}{\pgfpoint{24.626904\du}{15.000000\du}}{\pgfpoint{24.166667\du}{15.000000\du}}
\pgfpathlineto{\pgfpoint{20.833333\du}{15.000000\du}}
\pgfpathcurveto{\pgfpoint{20.373096\du}{15.000000\du}}{\pgfpoint{20.000000\du}{14.552285\du}}{\pgfpoint{20.000000\du}{14.000000\du}}
\pgfpathcurveto{\pgfpoint{20.000000\du}{13.447715\du}}{\pgfpoint{20.373096\du}{13.000000\du}}{\pgfpoint{20.833333\du}{13.000000\du}}
\pgfusepath{fill}
\definecolor{dialinecolor}{rgb}{0.000000, 0.000000, 0.000000}
\pgfsetstrokecolor{dialinecolor}
\pgfpathmoveto{\pgfpoint{20.833333\du}{13.000000\du}}
\pgfpathlineto{\pgfpoint{24.166667\du}{13.000000\du}}
\pgfpathcurveto{\pgfpoint{24.626904\du}{13.000000\du}}{\pgfpoint{25.000000\du}{13.447715\du}}{\pgfpoint{25.000000\du}{14.000000\du}}
\pgfpathcurveto{\pgfpoint{25.000000\du}{14.552285\du}}{\pgfpoint{24.626904\du}{15.000000\du}}{\pgfpoint{24.166667\du}{15.000000\du}}
\pgfpathlineto{\pgfpoint{20.833333\du}{15.000000\du}}
\pgfpathcurveto{\pgfpoint{20.373096\du}{15.000000\du}}{\pgfpoint{20.000000\du}{14.552285\du}}{\pgfpoint{20.000000\du}{14.000000\du}}
\pgfpathcurveto{\pgfpoint{20.000000\du}{13.447715\du}}{\pgfpoint{20.373096\du}{13.000000\du}}{\pgfpoint{20.833333\du}{13.000000\du}}
\pgfusepath{stroke}
% setfont left to latex
\definecolor{dialinecolor}{rgb}{0.000000, 0.000000, 0.000000}
\pgfsetstrokecolor{dialinecolor}
\node at (22.500000\du,14.200000\du){32bit R};
\pgfsetlinewidth{0.100000\du}
\pgfsetdash{}{0pt}
\pgfsetdash{}{0pt}
\pgfsetbuttcap
\pgfsetmiterjoin
\pgfsetlinewidth{0.100000\du}
\pgfsetbuttcap
\pgfsetmiterjoin
\pgfsetdash{}{0pt}
\definecolor{dialinecolor}{rgb}{1.000000, 1.000000, 1.000000}
\pgfsetfillcolor{dialinecolor}
\pgfpathmoveto{\pgfpoint{20.228125\du}{18.000000\du}}
\pgfpathlineto{\pgfpoint{25.140625\du}{18.000000\du}}
\pgfpathcurveto{\pgfpoint{25.818900\du}{18.000000\du}}{\pgfpoint{26.368750\du}{18.447715\du}}{\pgfpoint{26.368750\du}{19.000000\du}}
\pgfpathcurveto{\pgfpoint{26.368750\du}{19.552285\du}}{\pgfpoint{25.818900\du}{20.000000\du}}{\pgfpoint{25.140625\du}{20.000000\du}}
\pgfpathlineto{\pgfpoint{20.228125\du}{20.000000\du}}
\pgfpathcurveto{\pgfpoint{19.549850\du}{20.000000\du}}{\pgfpoint{19.000000\du}{19.552285\du}}{\pgfpoint{19.000000\du}{19.000000\du}}
\pgfpathcurveto{\pgfpoint{19.000000\du}{18.447715\du}}{\pgfpoint{19.549850\du}{18.000000\du}}{\pgfpoint{20.228125\du}{18.000000\du}}
\pgfusepath{fill}
\definecolor{dialinecolor}{rgb}{0.000000, 0.000000, 0.000000}
\pgfsetstrokecolor{dialinecolor}
\pgfpathmoveto{\pgfpoint{20.228125\du}{18.000000\du}}
\pgfpathlineto{\pgfpoint{25.140625\du}{18.000000\du}}
\pgfpathcurveto{\pgfpoint{25.818900\du}{18.000000\du}}{\pgfpoint{26.368750\du}{18.447715\du}}{\pgfpoint{26.368750\du}{19.000000\du}}
\pgfpathcurveto{\pgfpoint{26.368750\du}{19.552285\du}}{\pgfpoint{25.818900\du}{20.000000\du}}{\pgfpoint{25.140625\du}{20.000000\du}}
\pgfpathlineto{\pgfpoint{20.228125\du}{20.000000\du}}
\pgfpathcurveto{\pgfpoint{19.549850\du}{20.000000\du}}{\pgfpoint{19.000000\du}{19.552285\du}}{\pgfpoint{19.000000\du}{19.000000\du}}
\pgfpathcurveto{\pgfpoint{19.000000\du}{18.447715\du}}{\pgfpoint{19.549850\du}{18.000000\du}}{\pgfpoint{20.228125\du}{18.000000\du}}
\pgfusepath{stroke}
% setfont left to latex
\definecolor{dialinecolor}{rgb}{0.000000, 0.000000, 0.000000}
\pgfsetstrokecolor{dialinecolor}
\node at (22.684375\du,19.200000\du){"ODD"/"EVEN"};
\end{tikzpicture}
}
\caption{Description of an Odd-or-Even session}
\label{Fig: Odd or Even}
\end{figure}

CLIENT randomly generates a 32-bit unsigned integer R and sends it to SERVER in binary. SERVER replies with a string “ODD'' or “EVEN” according to the value of the 32-bit $R$(\Cref{Fig: Odd or Even}).

\subsection{Analysis}
We run the application for multiple times and collected the packets it generated. As we have expected, “ODD” packets are $1$ byte shorter than “EVEN” packets which implies that an eavesdropping adversary can learn what has been sent from SERVER to CLIENT simply by looking at the packets length. However, no obvious leakage has been found in other fields of the packets.

%For every \textbf{Odd-or-Even} session, 
%
%Packets from CLIENT to SERVER:
%
%All fields for every packet are the same, except:
%1. Encrypted Application Data field in DTLS layer.
%2. Sequence Number increased by 1 every packet.
%3. Checksum in UDP layer.
%
%Packets from SERVER to CLIENT:
%
%All fields are the same for every packet except:
%1. Encrypted Application Data field in DTLS layer.
%2. Sequence Number increased by 1 every packet.
%3. Checksum in UDP layer.
%4. Length field in both DTLS layer and UDP layer. The values are always (20,41) respectively when data is "Odd" and (21,42) when data is "Even".
%
%Therefore in this application, given pre-knowledge that server responds with either "Odd" or "Even", the length field in both DTLS layer and UDP layer can directly leak the plaintext. 

\section{Leaky Coffee}
\label{Sec: Leaky Coffee}

\subsection{Application Description}
\textbf{Leaky Coffee} simulates a more complicated scenario where the CLIENT sends a coffee order (in string) to SERVER. SERVER echoes the order appended by some flavour (in string). CLIENT compares the amount of given flavour to an internally generated random requirement and asks SERVER again for more additive if it is insufficient.

\begin{figure}[H]
\centering
\resizebox{12cm}{!}
{% Graphic for TeX using PGF
% Title: /home/yy12135/Google Drive/Writings/First-year-review_20150422/Pics/LeakyCoffee.dia
% Creator: Dia v0.97.2
% CreationDate: Tue Mar  3 14:05:39 2015
% For: yy12135
% \usepackage{tikz}
% The following commands are not supported in PSTricks at present
% We define them conditionally, so when they are implemented,
% this pgf file will use them.
\ifx\du\undefined
  \newlength{\du}
\fi
\setlength{\du}{15\unitlength}
\begin{tikzpicture}
\pgftransformxscale{1.000000}
\pgftransformyscale{-1.000000}
\definecolor{dialinecolor}{rgb}{0.000000, 0.000000, 0.000000}
\pgfsetstrokecolor{dialinecolor}
\definecolor{dialinecolor}{rgb}{1.000000, 1.000000, 1.000000}
\pgfsetfillcolor{dialinecolor}
\definecolor{dialinecolor}{rgb}{1.000000, 1.000000, 1.000000}
\pgfsetfillcolor{dialinecolor}
\pgfpathellipse{\pgfpoint{10.000000\du}{11.000000\du}}{\pgfpoint{1.723870\du}{0\du}}{\pgfpoint{0\du}{1.723870\du}}
\pgfusepath{fill}
\pgfsetlinewidth{0.100000\du}
\pgfsetdash{}{0pt}
\pgfsetdash{}{0pt}
\pgfsetmiterjoin
\definecolor{dialinecolor}{rgb}{0.000000, 0.000000, 0.000000}
\pgfsetstrokecolor{dialinecolor}
\pgfpathellipse{\pgfpoint{10.000000\du}{11.000000\du}}{\pgfpoint{1.723870\du}{0\du}}{\pgfpoint{0\du}{1.723870\du}}
\pgfusepath{stroke}
% setfont left to latex
\definecolor{dialinecolor}{rgb}{0.000000, 0.000000, 0.000000}
\pgfsetstrokecolor{dialinecolor}
\node at (10.000000\du,11.195000\du){CLIENT};
% setfont left to latex
\definecolor{dialinecolor}{rgb}{0.000000, 0.000000, 0.000000}
\pgfsetstrokecolor{dialinecolor}
\node[anchor=west] at (10.000000\du,11.000000\du){};
\definecolor{dialinecolor}{rgb}{1.000000, 1.000000, 1.000000}
\pgfsetfillcolor{dialinecolor}
\pgfpathellipse{\pgfpoint{25.008494\du}{10.996126\du}}{\pgfpoint{1.836494\du}{0\du}}{\pgfpoint{0\du}{1.813886\du}}
\pgfusepath{fill}
\pgfsetlinewidth{0.100000\du}
\pgfsetdash{}{0pt}
\pgfsetdash{}{0pt}
\pgfsetmiterjoin
\definecolor{dialinecolor}{rgb}{0.000000, 0.000000, 0.000000}
\pgfsetstrokecolor{dialinecolor}
\pgfpathellipse{\pgfpoint{25.008494\du}{10.996126\du}}{\pgfpoint{1.836494\du}{0\du}}{\pgfpoint{0\du}{1.813886\du}}
\pgfusepath{stroke}
% setfont left to latex
\definecolor{dialinecolor}{rgb}{0.000000, 0.000000, 0.000000}
\pgfsetstrokecolor{dialinecolor}
\node at (25.008494\du,11.191126\du){SERVER};
\pgfsetlinewidth{0.100000\du}
\pgfsetdash{}{0pt}
\pgfsetdash{}{0pt}
\pgfsetbuttcap
{
\definecolor{dialinecolor}{rgb}{0.000000, 0.000000, 0.000000}
\pgfsetfillcolor{dialinecolor}
% was here!!!
\pgfsetarrowsend{to}
\definecolor{dialinecolor}{rgb}{0.000000, 0.000000, 0.000000}
\pgfsetstrokecolor{dialinecolor}
\draw (10.000000\du,12.774002\du)--(10.000000\du,40.000000\du);
}
\pgfsetlinewidth{0.100000\du}
\pgfsetdash{}{0pt}
\pgfsetdash{}{0pt}
\pgfsetbuttcap
{
\definecolor{dialinecolor}{rgb}{0.000000, 0.000000, 0.000000}
\pgfsetfillcolor{dialinecolor}
% was here!!!
\pgfsetarrowsend{to}
\definecolor{dialinecolor}{rgb}{0.000000, 0.000000, 0.000000}
\pgfsetstrokecolor{dialinecolor}
\draw (25.007949\du,12.859321\du)--(25.000000\du,40.000000\du);
}
\pgfsetlinewidth{0.100000\du}
\pgfsetdash{}{0pt}
\pgfsetdash{}{0pt}
\pgfsetbuttcap
{
\definecolor{dialinecolor}{rgb}{0.000000, 0.000000, 0.000000}
\pgfsetfillcolor{dialinecolor}
% was here!!!
\pgfsetarrowsend{to}
\definecolor{dialinecolor}{rgb}{0.000000, 0.000000, 0.000000}
\pgfsetstrokecolor{dialinecolor}
\draw (10.000000\du,15.799100\du)--(25.006200\du,18.890600\du);
}
\pgfsetlinewidth{0.100000\du}
\pgfsetdash{}{0pt}
\pgfsetdash{}{0pt}
\pgfsetbuttcap
{
\definecolor{dialinecolor}{rgb}{0.000000, 0.000000, 0.000000}
\pgfsetfillcolor{dialinecolor}
% was here!!!
\pgfsetarrowsend{to}
\definecolor{dialinecolor}{rgb}{0.000000, 0.000000, 0.000000}
\pgfsetstrokecolor{dialinecolor}
\draw (25.005300\du,21.906200\du)--(10.000000\du,24.874400\du);
}
\pgfsetlinewidth{0.100000\du}
\pgfsetdash{}{0pt}
\pgfsetdash{}{0pt}
\pgfsetbuttcap
\pgfsetmiterjoin
\pgfsetlinewidth{0.100000\du}
\pgfsetbuttcap
\pgfsetmiterjoin
\pgfsetdash{}{0pt}
\definecolor{dialinecolor}{rgb}{1.000000, 1.000000, 1.000000}
\pgfsetfillcolor{dialinecolor}
\pgfpathmoveto{\pgfpoint{13.558175\du}{16.000000\du}}
\pgfpathlineto{\pgfpoint{19.625675\du}{16.000000\du}}
\pgfpathcurveto{\pgfpoint{20.463422\du}{16.000000\du}}{\pgfpoint{21.142550\du}{16.447715\du}}{\pgfpoint{21.142550\du}{17.000000\du}}
\pgfpathcurveto{\pgfpoint{21.142550\du}{17.552285\du}}{\pgfpoint{20.463422\du}{18.000000\du}}{\pgfpoint{19.625675\du}{18.000000\du}}
\pgfpathlineto{\pgfpoint{13.558175\du}{18.000000\du}}
\pgfpathcurveto{\pgfpoint{12.720428\du}{18.000000\du}}{\pgfpoint{12.041300\du}{17.552285\du}}{\pgfpoint{12.041300\du}{17.000000\du}}
\pgfpathcurveto{\pgfpoint{12.041300\du}{16.447715\du}}{\pgfpoint{12.720428\du}{16.000000\du}}{\pgfpoint{13.558175\du}{16.000000\du}}
\pgfusepath{fill}
\definecolor{dialinecolor}{rgb}{0.000000, 0.000000, 0.000000}
\pgfsetstrokecolor{dialinecolor}
\pgfpathmoveto{\pgfpoint{13.558175\du}{16.000000\du}}
\pgfpathlineto{\pgfpoint{19.625675\du}{16.000000\du}}
\pgfpathcurveto{\pgfpoint{20.463422\du}{16.000000\du}}{\pgfpoint{21.142550\du}{16.447715\du}}{\pgfpoint{21.142550\du}{17.000000\du}}
\pgfpathcurveto{\pgfpoint{21.142550\du}{17.552285\du}}{\pgfpoint{20.463422\du}{18.000000\du}}{\pgfpoint{19.625675\du}{18.000000\du}}
\pgfpathlineto{\pgfpoint{13.558175\du}{18.000000\du}}
\pgfpathcurveto{\pgfpoint{12.720428\du}{18.000000\du}}{\pgfpoint{12.041300\du}{17.552285\du}}{\pgfpoint{12.041300\du}{17.000000\du}}
\pgfpathcurveto{\pgfpoint{12.041300\du}{16.447715\du}}{\pgfpoint{12.720428\du}{16.000000\du}}{\pgfpoint{13.558175\du}{16.000000\du}}
\pgfusepath{stroke}
% setfont left to latex
\definecolor{dialinecolor}{rgb}{0.000000, 0.000000, 0.000000}
\pgfsetstrokecolor{dialinecolor}
\node at (16.591925\du,17.200000\du){\textbf{1}. $Order$};
\pgfsetlinewidth{0.100000\du}
\pgfsetdash{}{0pt}
\pgfsetdash{}{0pt}
\pgfsetbuttcap
\pgfsetmiterjoin
\pgfsetlinewidth{0.100000\du}
\pgfsetbuttcap
\pgfsetmiterjoin
\pgfsetdash{}{0pt}
\definecolor{dialinecolor}{rgb}{1.000000, 1.000000, 1.000000}
\pgfsetfillcolor{dialinecolor}
\pgfpathmoveto{\pgfpoint{12.990000\du}{22.000000\du}}
\pgfpathlineto{\pgfpoint{20.950000\du}{22.000000\du}}
\pgfpathcurveto{\pgfpoint{22.049047\du}{22.000000\du}}{\pgfpoint{22.940000\du}{22.447715\du}}{\pgfpoint{22.940000\du}{23.000000\du}}
\pgfpathcurveto{\pgfpoint{22.940000\du}{23.552285\du}}{\pgfpoint{22.049047\du}{24.000000\du}}{\pgfpoint{20.950000\du}{24.000000\du}}
\pgfpathlineto{\pgfpoint{12.990000\du}{24.000000\du}}
\pgfpathcurveto{\pgfpoint{11.890953\du}{24.000000\du}}{\pgfpoint{11.000000\du}{23.552285\du}}{\pgfpoint{11.000000\du}{23.000000\du}}
\pgfpathcurveto{\pgfpoint{11.000000\du}{22.447715\du}}{\pgfpoint{11.890953\du}{22.000000\du}}{\pgfpoint{12.990000\du}{22.000000\du}}
\pgfusepath{fill}
\definecolor{dialinecolor}{rgb}{0.000000, 0.000000, 0.000000}
\pgfsetstrokecolor{dialinecolor}
\pgfpathmoveto{\pgfpoint{12.990000\du}{22.000000\du}}
\pgfpathlineto{\pgfpoint{20.950000\du}{22.000000\du}}
\pgfpathcurveto{\pgfpoint{22.049047\du}{22.000000\du}}{\pgfpoint{22.940000\du}{22.447715\du}}{\pgfpoint{22.940000\du}{23.000000\du}}
\pgfpathcurveto{\pgfpoint{22.940000\du}{23.552285\du}}{\pgfpoint{22.049047\du}{24.000000\du}}{\pgfpoint{20.950000\du}{24.000000\du}}
\pgfpathlineto{\pgfpoint{12.990000\du}{24.000000\du}}
\pgfpathcurveto{\pgfpoint{11.890953\du}{24.000000\du}}{\pgfpoint{11.000000\du}{23.552285\du}}{\pgfpoint{11.000000\du}{23.000000\du}}
\pgfpathcurveto{\pgfpoint{11.000000\du}{22.447715\du}}{\pgfpoint{11.890953\du}{22.000000\du}}{\pgfpoint{12.990000\du}{22.000000\du}}
\pgfusepath{stroke}
% setfont left to latex
\definecolor{dialinecolor}{rgb}{0.000000, 0.000000, 0.000000}
\pgfsetstrokecolor{dialinecolor}
\node at (16.970000\du,23.200000\du){\textbf{2}.$Order || Flavour$};
\pgfsetlinewidth{0.100000\du}
\pgfsetdash{{1.000000\du}{1.000000\du}}{0\du}
\pgfsetdash{{1.000000\du}{1.000000\du}}{0\du}
\pgfsetbuttcap
{
\definecolor{dialinecolor}{rgb}{0.000000, 0.000000, 0.000000}
\pgfsetfillcolor{dialinecolor}
% was here!!!
\definecolor{dialinecolor}{rgb}{0.000000, 0.000000, 0.000000}
\pgfsetstrokecolor{dialinecolor}
\draw (9.010000\du,25.950000\du)--(36.000000\du,26.000000\du);
}
\definecolor{dialinecolor}{rgb}{1.000000, 1.000000, 1.000000}
\pgfsetfillcolor{dialinecolor}
\fill (0.000000\du,25.000000\du)--(0.000000\du,26.900000\du)--(9.010000\du,26.900000\du)--(9.010000\du,25.000000\du)--cycle;
\pgfsetlinewidth{0.100000\du}
\pgfsetdash{}{0pt}
\pgfsetdash{}{0pt}
\pgfsetmiterjoin
\definecolor{dialinecolor}{rgb}{0.000000, 0.000000, 0.000000}
\pgfsetstrokecolor{dialinecolor}
\draw (0.000000\du,25.000000\du)--(0.000000\du,26.900000\du)--(9.010000\du,26.900000\du)--(9.010000\du,25.000000\du)--cycle;
% setfont left to latex
\definecolor{dialinecolor}{rgb}{0.000000, 0.000000, 0.000000}
\pgfsetstrokecolor{dialinecolor}
\node at (4.505000\du,26.145000\du){If $Flavour$ is not enough};
\pgfsetlinewidth{0.100000\du}
\pgfsetdash{}{0pt}
\pgfsetdash{}{0pt}
\pgfsetbuttcap
{
\definecolor{dialinecolor}{rgb}{0.000000, 0.000000, 0.000000}
\pgfsetfillcolor{dialinecolor}
% was here!!!
\pgfsetarrowsend{to}
\definecolor{dialinecolor}{rgb}{0.000000, 0.000000, 0.000000}
\pgfsetstrokecolor{dialinecolor}
\draw (10.000000\du,27.899600\du)--(25.002700\du,30.953100\du);
}
\pgfsetlinewidth{0.100000\du}
\pgfsetdash{}{0pt}
\pgfsetdash{}{0pt}
\pgfsetbuttcap
\pgfsetmiterjoin
\pgfsetlinewidth{0.100000\du}
\pgfsetbuttcap
\pgfsetmiterjoin
\pgfsetdash{}{0pt}
\definecolor{dialinecolor}{rgb}{1.000000, 1.000000, 1.000000}
\pgfsetfillcolor{dialinecolor}
\pgfpathmoveto{\pgfpoint{14.065625\du}{28.000000\du}}
\pgfpathlineto{\pgfpoint{20.158125\du}{28.000000\du}}
\pgfpathcurveto{\pgfpoint{20.999324\du}{28.000000\du}}{\pgfpoint{21.681250\du}{28.447715\du}}{\pgfpoint{21.681250\du}{29.000000\du}}
\pgfpathcurveto{\pgfpoint{21.681250\du}{29.552285\du}}{\pgfpoint{20.999324\du}{30.000000\du}}{\pgfpoint{20.158125\du}{30.000000\du}}
\pgfpathlineto{\pgfpoint{14.065625\du}{30.000000\du}}
\pgfpathcurveto{\pgfpoint{13.224426\du}{30.000000\du}}{\pgfpoint{12.542500\du}{29.552285\du}}{\pgfpoint{12.542500\du}{29.000000\du}}
\pgfpathcurveto{\pgfpoint{12.542500\du}{28.447715\du}}{\pgfpoint{13.224426\du}{28.000000\du}}{\pgfpoint{14.065625\du}{28.000000\du}}
\pgfusepath{fill}
\definecolor{dialinecolor}{rgb}{0.000000, 0.000000, 0.000000}
\pgfsetstrokecolor{dialinecolor}
\pgfpathmoveto{\pgfpoint{14.065625\du}{28.000000\du}}
\pgfpathlineto{\pgfpoint{20.158125\du}{28.000000\du}}
\pgfpathcurveto{\pgfpoint{20.999324\du}{28.000000\du}}{\pgfpoint{21.681250\du}{28.447715\du}}{\pgfpoint{21.681250\du}{29.000000\du}}
\pgfpathcurveto{\pgfpoint{21.681250\du}{29.552285\du}}{\pgfpoint{20.999324\du}{30.000000\du}}{\pgfpoint{20.158125\du}{30.000000\du}}
\pgfpathlineto{\pgfpoint{14.065625\du}{30.000000\du}}
\pgfpathcurveto{\pgfpoint{13.224426\du}{30.000000\du}}{\pgfpoint{12.542500\du}{29.552285\du}}{\pgfpoint{12.542500\du}{29.000000\du}}
\pgfpathcurveto{\pgfpoint{12.542500\du}{28.447715\du}}{\pgfpoint{13.224426\du}{28.000000\du}}{\pgfpoint{14.065625\du}{28.000000\du}}
\pgfusepath{stroke}
% setfont left to latex
\definecolor{dialinecolor}{rgb}{0.000000, 0.000000, 0.000000}
\pgfsetstrokecolor{dialinecolor}
\node at (17.111875\du,29.200000\du){\textbf{3}.$FlavourRequest$};
\pgfsetlinewidth{0.100000\du}
\pgfsetdash{{1.000000\du}{1.000000\du}}{0\du}
\pgfsetdash{{1.000000\du}{1.000000\du}}{0\du}
\pgfsetbuttcap
{
\definecolor{dialinecolor}{rgb}{0.000000, 0.000000, 0.000000}
\pgfsetfillcolor{dialinecolor}
% was here!!!
\definecolor{dialinecolor}{rgb}{0.000000, 0.000000, 0.000000}
\pgfsetstrokecolor{dialinecolor}
\draw (1.000000\du,38.000000\du)--(36.000000\du,38.000000\du);
}
\pgfsetlinewidth{0.100000\du}
\pgfsetdash{}{0pt}
\pgfsetdash{}{0pt}
\pgfsetbuttcap
{
\definecolor{dialinecolor}{rgb}{0.000000, 0.000000, 0.000000}
\pgfsetfillcolor{dialinecolor}
% was here!!!
\pgfsetarrowsend{to}
\definecolor{dialinecolor}{rgb}{0.000000, 0.000000, 0.000000}
\pgfsetstrokecolor{dialinecolor}
\draw (25.001800\du,33.968700\du)--(10.000000\du,36.974900\du);
}
\pgfsetlinewidth{0.100000\du}
\pgfsetdash{}{0pt}
\pgfsetdash{}{0pt}
\pgfsetbuttcap
\pgfsetmiterjoin
\pgfsetlinewidth{0.100000\du}
\pgfsetbuttcap
\pgfsetmiterjoin
\pgfsetdash{}{0pt}
\definecolor{dialinecolor}{rgb}{1.000000, 1.000000, 1.000000}
\pgfsetfillcolor{dialinecolor}
\pgfpathmoveto{\pgfpoint{13.779375\du}{34.000000\du}}
\pgfpathlineto{\pgfpoint{20.236875\du}{34.000000\du}}
\pgfpathcurveto{\pgfpoint{21.128470\du}{34.000000\du}}{\pgfpoint{21.851250\du}{34.447715\du}}{\pgfpoint{21.851250\du}{35.000000\du}}
\pgfpathcurveto{\pgfpoint{21.851250\du}{35.552285\du}}{\pgfpoint{21.128470\du}{36.000000\du}}{\pgfpoint{20.236875\du}{36.000000\du}}
\pgfpathlineto{\pgfpoint{13.779375\du}{36.000000\du}}
\pgfpathcurveto{\pgfpoint{12.887780\du}{36.000000\du}}{\pgfpoint{12.165000\du}{35.552285\du}}{\pgfpoint{12.165000\du}{35.000000\du}}
\pgfpathcurveto{\pgfpoint{12.165000\du}{34.447715\du}}{\pgfpoint{12.887780\du}{34.000000\du}}{\pgfpoint{13.779375\du}{34.000000\du}}
\pgfusepath{fill}
\definecolor{dialinecolor}{rgb}{0.000000, 0.000000, 0.000000}
\pgfsetstrokecolor{dialinecolor}
\pgfpathmoveto{\pgfpoint{13.779375\du}{34.000000\du}}
\pgfpathlineto{\pgfpoint{20.236875\du}{34.000000\du}}
\pgfpathcurveto{\pgfpoint{21.128470\du}{34.000000\du}}{\pgfpoint{21.851250\du}{34.447715\du}}{\pgfpoint{21.851250\du}{35.000000\du}}
\pgfpathcurveto{\pgfpoint{21.851250\du}{35.552285\du}}{\pgfpoint{21.128470\du}{36.000000\du}}{\pgfpoint{20.236875\du}{36.000000\du}}
\pgfpathlineto{\pgfpoint{13.779375\du}{36.000000\du}}
\pgfpathcurveto{\pgfpoint{12.887780\du}{36.000000\du}}{\pgfpoint{12.165000\du}{35.552285\du}}{\pgfpoint{12.165000\du}{35.000000\du}}
\pgfpathcurveto{\pgfpoint{12.165000\du}{34.447715\du}}{\pgfpoint{12.887780\du}{34.000000\du}}{\pgfpoint{13.779375\du}{34.000000\du}}
\pgfusepath{stroke}
% setfont left to latex
\definecolor{dialinecolor}{rgb}{0.000000, 0.000000, 0.000000}
\pgfsetstrokecolor{dialinecolor}
\node at (17.008125\du,35.200000\du){\textbf{4}.$FlavourResponse$};
\pgfsetlinewidth{0.100000\du}
\pgfsetdash{{1.000000\du}{1.000000\du}}{0\du}
\pgfsetdash{{1.000000\du}{1.000000\du}}{0\du}
\pgfsetbuttcap
{
\definecolor{dialinecolor}{rgb}{0.000000, 0.000000, 0.000000}
\pgfsetfillcolor{dialinecolor}
% was here!!!
\pgfsetarrowsstart{to}
\pgfsetarrowsend{to}
\definecolor{dialinecolor}{rgb}{0.000000, 0.000000, 0.000000}
\pgfsetstrokecolor{dialinecolor}
\draw (5.000000\du,27.000000\du)--(5.000000\du,38.000000\du);
}
\end{tikzpicture}
}
\caption{Description of a \textbf{Leaky Coffee} session}
\label{Fig: Description of a Leaky Coffee session}
\end{figure}

\Cref{Fig: Description of a Leaky Coffee session} describes the procedure of a \textbf{Leaky Coffee} session. \Cref{Ex: An example with $FlavourRequest$ and $FlavourResponse$} and \Cref{Ex: A Leaky-Coffee session without $FlavourRequest$ and $FlavourResponse$} are two examples for a 2 packets and a 4 packets session respectively.

\subsubsection{Syntaxes}
“$A||B$” represents “String $A$ concatenated by string $B$”. 

“$|A|$” represents the length of string $A$.

\begin{definition} \label{Def: Order}
$Order$ is an ASCII string randomly selected as:
\begin{equation*}
Order = \text{“AMERICANO”} | \text{“CAPPUCCINO”} | \text{“ESPRESSO”} | \text{“MOCHA”}
\end{equation*}
\end{definition}

\begin{definition} \label{Def: Coffee}
$Coffee$ is an ASCII string constructed by three substrings: $Order || Milk || Sugar$. $Order$ is defined in \Cref{Def: Order}. $Milk$ and $Sugar$ are composed of no more than 3 ‘@’ and ‘*’ respectively:\\
\begin{equation*}
\begin{aligned}
Coffee &= Order || Milk || Sugar\\
Milk &= \{\text{‘@’}\}^{\{0,3\}}\\
Sugar &= \{\text{‘*’}\}^{\{0,3\}}\\
\end{aligned}
\end{equation*}
\end{definition}

\begin{definition} \label{Def: FlavourRequest}
$FlavourRequest$ is an ASCII string begins with “FLAVOUR” and followed by a $Milk$ then a $Sugar$ defined in \Cref{Def: Coffee}.
\begin{equation*}
FlavourRequest = \text{“FLAVOUR”} || Milk || Sugar
\end{equation*}
\end{definition}

\begin{definition} \label{Def: FlavourResponse}
$FlavourResponse$ is identical to $FlavourRequest$ defined in \Cref{Def: FlavourRequest}.
\begin{equation*}
FlavourReponse = FlavourRequest =  \text{“FLAVOUR”} || Milk || Sugar
\end{equation*}
\end{definition}

\begin{definition} \label{Def: Leaky Coffee Session}
\subsubsection{Leaky Coffee Session}
A \textbf{Leaky Coffee} session performs under the following procedure:

\begin{description}
\item[1.] CLIENT randomly picks an $Order$(\Cref{Def: Order}) and sends it to SERVER.

\item[2.] SERVER replies to CLIENT with a $Coffee$(\Cref{Def: Coffee}) where the first part is identical to the $Order$ received. If the $Order$ is “ESPRESSO” then $Milk$ and $Sugar$ are set to be NULL string; otherwise they are generated randomly.

\item[3.] If $Coffee$ is “ESPRESSO” then the session is completed; otherwise CLIENT compares $|Milk|$ and $|Sugar|$ with two random integer selected from $[0,3]$ by itself. If any of them are smaller than the integer, then CLIENT sends out a $FlavourRequest$(\Cref{Def: FlavourRequest}) with its $Milk$ and $Sugar$ corresponds to the insufficient part.

\item[4.] Upon receving a $FlavourRequest$, SERVER sends back a $FlavourResponse$(\Cref{Def: FlavourResponse}) that is identical to the $FlavourRequest$ it received.

\item[5.] CLIENT randomly sleeps for $5$ to $10$ seconds before re-initiates another \textbf{Leaky Coffee} session.
\end{description}
In current implementation, all random values are generated \textbf{uniformly}.
\end{definition}

\subsubsection{Leaky Coffee session examples}
As in \Cref{Def: Leaky Coffee Session}, $FlavourRequest$ and $FlavourResponse$ only appears when $Sugar$ and/or $Milk$ are insufficient in $Coffee$; therefore \textbf{Leaky Coffee} sessions can be categorised by the existence of $FlavourRequest$ and $FlavourResponse$.

\begin{example} \label{Ex: An example with $FlavourRequest$ and $FlavourResponse$}
{An example with $FlavourRequest$ and $FlavourResponse$(\Cref{Fig: Leaky-Coffee Example1})}

{
\begin{figure}[H]
\centering
\resizebox{7cm}{!}
{% Graphic for TeX using PGF
% Title: /home/yy12135/Writings/First-year-review_20150422/Pics/LeakyCoffee_example1.dia
% Creator: Dia v0.97.2
% CreationDate: Tue Mar  3 11:33:54 2015
% For: yy12135
% \usepackage{tikz}
% The following commands are not supported in PSTricks at present
% We define them conditionally, so when they are implemented,
% this pgf file will use them.
\ifx\du\undefined
  \newlength{\du}
\fi
\setlength{\du}{15\unitlength}
\begin{tikzpicture}
\pgftransformxscale{1.000000}
\pgftransformyscale{-1.000000}
\definecolor{dialinecolor}{rgb}{0.000000, 0.000000, 0.000000}
\pgfsetstrokecolor{dialinecolor}
\definecolor{dialinecolor}{rgb}{1.000000, 1.000000, 1.000000}
\pgfsetfillcolor{dialinecolor}
\definecolor{dialinecolor}{rgb}{1.000000, 1.000000, 1.000000}
\pgfsetfillcolor{dialinecolor}
\pgfpathellipse{\pgfpoint{27.000000\du}{12.000000\du}}{\pgfpoint{2.000000\du}{0\du}}{\pgfpoint{0\du}{2.000000\du}}
\pgfusepath{fill}
\pgfsetlinewidth{0.100000\du}
\pgfsetdash{}{0pt}
\pgfsetdash{}{0pt}
\pgfsetmiterjoin
\definecolor{dialinecolor}{rgb}{0.000000, 0.000000, 0.000000}
\pgfsetstrokecolor{dialinecolor}
\pgfpathellipse{\pgfpoint{27.000000\du}{12.000000\du}}{\pgfpoint{2.000000\du}{0\du}}{\pgfpoint{0\du}{2.000000\du}}
\pgfusepath{stroke}
% setfont left to latex
\definecolor{dialinecolor}{rgb}{0.000000, 0.000000, 0.000000}
\pgfsetstrokecolor{dialinecolor}
\node at (27.000000\du,12.195000\du){CLIENT};
\definecolor{dialinecolor}{rgb}{1.000000, 1.000000, 1.000000}
\pgfsetfillcolor{dialinecolor}
\pgfpathellipse{\pgfpoint{40.051131\du}{12.044456\du}}{\pgfpoint{2.051131\du}{0\du}}{\pgfpoint{0\du}{2.044456\du}}
\pgfusepath{fill}
\pgfsetlinewidth{0.100000\du}
\pgfsetdash{}{0pt}
\pgfsetdash{}{0pt}
\pgfsetmiterjoin
\definecolor{dialinecolor}{rgb}{0.000000, 0.000000, 0.000000}
\pgfsetstrokecolor{dialinecolor}
\pgfpathellipse{\pgfpoint{40.051131\du}{12.044456\du}}{\pgfpoint{2.051131\du}{0\du}}{\pgfpoint{0\du}{2.044456\du}}
\pgfusepath{stroke}
% setfont left to latex
\definecolor{dialinecolor}{rgb}{0.000000, 0.000000, 0.000000}
\pgfsetstrokecolor{dialinecolor}
\node at (40.051131\du,12.239456\du){SERVER};
\pgfsetlinewidth{0.100000\du}
\pgfsetdash{}{0pt}
\pgfsetdash{}{0pt}
\pgfsetbuttcap
{
\definecolor{dialinecolor}{rgb}{0.000000, 0.000000, 0.000000}
\pgfsetfillcolor{dialinecolor}
% was here!!!
\pgfsetarrowsend{to}
\definecolor{dialinecolor}{rgb}{0.000000, 0.000000, 0.000000}
\pgfsetstrokecolor{dialinecolor}
\draw (27.000000\du,14.000000\du)--(27.000000\du,40.000000\du);
}
\pgfsetlinewidth{0.100000\du}
\pgfsetdash{}{0pt}
\pgfsetdash{}{0pt}
\pgfsetbuttcap
{
\definecolor{dialinecolor}{rgb}{0.000000, 0.000000, 0.000000}
\pgfsetfillcolor{dialinecolor}
% was here!!!
\pgfsetarrowsend{to}
\definecolor{dialinecolor}{rgb}{0.000000, 0.000000, 0.000000}
\pgfsetstrokecolor{dialinecolor}
\draw (40.051131\du,14.088911\du)--(40.000000\du,40.000000\du);
}
\pgfsetlinewidth{0.100000\du}
\pgfsetdash{}{0pt}
\pgfsetdash{}{0pt}
\pgfsetbuttcap
{
\definecolor{dialinecolor}{rgb}{0.000000, 0.000000, 0.000000}
\pgfsetfillcolor{dialinecolor}
% was here!!!
\pgfsetarrowsend{to}
\definecolor{dialinecolor}{rgb}{0.000000, 0.000000, 0.000000}
\pgfsetstrokecolor{dialinecolor}
\draw (27.000000\du,16.888889\du)--(40.128953\du,20.642722\du);
}
\pgfsetlinewidth{0.100000\du}
\pgfsetdash{}{0pt}
\pgfsetdash{}{0pt}
\pgfsetbuttcap
{
\definecolor{dialinecolor}{rgb}{0.000000, 0.000000, 0.000000}
\pgfsetfillcolor{dialinecolor}
% was here!!!
\pgfsetarrowsend{to}
\definecolor{dialinecolor}{rgb}{0.000000, 0.000000, 0.000000}
\pgfsetstrokecolor{dialinecolor}
\draw (40.034087\du,22.725941\du)--(27.000000\du,25.555556\du);
}
\pgfsetlinewidth{0.100000\du}
\pgfsetdash{}{0pt}
\pgfsetdash{}{0pt}
\pgfsetbuttcap
{
\definecolor{dialinecolor}{rgb}{0.000000, 0.000000, 0.000000}
\pgfsetfillcolor{dialinecolor}
% was here!!!
\pgfsetarrowsend{to}
\definecolor{dialinecolor}{rgb}{0.000000, 0.000000, 0.000000}
\pgfsetstrokecolor{dialinecolor}
\draw (27.000000\du,28.444444\du)--(40.017044\du,31.362970\du);
}
% setfont left to latex
\definecolor{dialinecolor}{rgb}{0.000000, 0.000000, 0.000000}
\pgfsetstrokecolor{dialinecolor}
\node[anchor=west] at (33.517044\du,24.140748\du){};
% setfont left to latex
\definecolor{dialinecolor}{rgb}{0.000000, 0.000000, 0.000000}
\pgfsetstrokecolor{dialinecolor}
\node[anchor=west] at (32.000000\du,20.000000\du){};
% setfont left to latex
\definecolor{dialinecolor}{rgb}{0.000000, 0.000000, 0.000000}
\pgfsetstrokecolor{dialinecolor}
\node[anchor=west] at (33.508522\du,29.903707\du){};
\pgfsetlinewidth{0.100000\du}
\pgfsetdash{}{0pt}
\pgfsetdash{}{0pt}
\pgfsetbuttcap
{
\definecolor{dialinecolor}{rgb}{0.000000, 0.000000, 0.000000}
\pgfsetfillcolor{dialinecolor}
% was here!!!
\pgfsetarrowsend{to}
\definecolor{dialinecolor}{rgb}{0.000000, 0.000000, 0.000000}
\pgfsetstrokecolor{dialinecolor}
\draw (40.011362\du,34.241980\du)--(27.000000\du,37.111111\du);
}
\definecolor{dialinecolor}{rgb}{1.000000, 1.000000, 1.000000}
\pgfsetfillcolor{dialinecolor}
\fill (31.351581\du,18.005049\du)--(31.351581\du,19.905049\du)--(35.466581\du,19.905049\du)--(35.466581\du,18.005049\du)--cycle;
\pgfsetlinewidth{0.100000\du}
\pgfsetdash{}{0pt}
\pgfsetdash{}{0pt}
\pgfsetmiterjoin
\definecolor{dialinecolor}{rgb}{0.000000, 0.000000, 0.000000}
\pgfsetstrokecolor{dialinecolor}
\draw (31.351581\du,18.005049\du)--(31.351581\du,19.905049\du)--(35.466581\du,19.905049\du)--(35.466581\du,18.005049\du)--cycle;
% setfont left to latex
\definecolor{dialinecolor}{rgb}{0.000000, 0.000000, 0.000000}
\pgfsetstrokecolor{dialinecolor}
\node at (33.409081\du,19.150049\du){"MOCHA"};
\definecolor{dialinecolor}{rgb}{1.000000, 1.000000, 1.000000}
\pgfsetfillcolor{dialinecolor}
\fill (30.842846\du,23.045766\du)--(30.842846\du,24.945766\du)--(35.917846\du,24.945766\du)--(35.917846\du,23.045766\du)--cycle;
\pgfsetlinewidth{0.100000\du}
\pgfsetdash{}{0pt}
\pgfsetdash{}{0pt}
\pgfsetmiterjoin
\definecolor{dialinecolor}{rgb}{0.000000, 0.000000, 0.000000}
\pgfsetstrokecolor{dialinecolor}
\draw (30.842846\du,23.045766\du)--(30.842846\du,24.945766\du)--(35.917846\du,24.945766\du)--(35.917846\du,23.045766\du)--cycle;
% setfont left to latex
\definecolor{dialinecolor}{rgb}{0.000000, 0.000000, 0.000000}
\pgfsetstrokecolor{dialinecolor}
\node at (33.380346\du,24.190766\du){"MOCHA*@"};
\definecolor{dialinecolor}{rgb}{1.000000, 1.000000, 1.000000}
\pgfsetfillcolor{dialinecolor}
\fill (30.418621\du,29.238167\du)--(30.418621\du,31.138167\du)--(36.731121\du,31.138167\du)--(36.731121\du,29.238167\du)--cycle;
\pgfsetlinewidth{0.100000\du}
\pgfsetdash{}{0pt}
\pgfsetdash{}{0pt}
\pgfsetmiterjoin
\definecolor{dialinecolor}{rgb}{0.000000, 0.000000, 0.000000}
\pgfsetstrokecolor{dialinecolor}
\draw (30.418621\du,29.238167\du)--(30.418621\du,31.138167\du)--(36.731121\du,31.138167\du)--(36.731121\du,29.238167\du)--cycle;
% setfont left to latex
\definecolor{dialinecolor}{rgb}{0.000000, 0.000000, 0.000000}
\pgfsetstrokecolor{dialinecolor}
\node at (33.574871\du,30.383167\du){"FLAVOUR**@"};
\definecolor{dialinecolor}{rgb}{1.000000, 1.000000, 1.000000}
\pgfsetfillcolor{dialinecolor}
\fill (30.598621\du,34.620045\du)--(30.598621\du,36.520045\du)--(36.551121\du,36.520045\du)--(36.551121\du,34.620045\du)--cycle;
\pgfsetlinewidth{0.100000\du}
\pgfsetdash{}{0pt}
\pgfsetdash{}{0pt}
\pgfsetmiterjoin
\definecolor{dialinecolor}{rgb}{0.000000, 0.000000, 0.000000}
\pgfsetstrokecolor{dialinecolor}
\draw (30.598621\du,34.620045\du)--(30.598621\du,36.520045\du)--(36.551121\du,36.520045\du)--(36.551121\du,34.620045\du)--cycle;
% setfont left to latex
\definecolor{dialinecolor}{rgb}{0.000000, 0.000000, 0.000000}
\pgfsetstrokecolor{dialinecolor}
\node at (33.574871\du,35.765045\du){"FLAVOUR**@"};
\end{tikzpicture}
}
\caption{Example: A \textbf{Leaky Coffee} session with $FlavourRequest$ and $FlavourResponse$}
\label{Fig: Leaky-Coffee Example1}
\end{figure}
}
\end{example}

\begin{example} \label{Ex: A Leaky-Coffee session without $FlavourRequest$ and $FlavourResponse$}
{Another example without $FlavourRequest$ and $FlavourResponse$(\Cref{Fig: Leaky-Coffee Example2})}:

\begin{figure}[H]
\centering
\resizebox{6cm}{!}
{% Graphic for TeX using PGF
% Title: /home/yy12135/Writings/First-year-review_20150422/Pics/LeakyCoffee_example2.dia
% Creator: Dia v0.97.2
% CreationDate: Tue Mar  3 13:37:23 2015
% For: yy12135
% \usepackage{tikz}
% The following commands are not supported in PSTricks at present
% We define them conditionally, so when they are implemented,
% this pgf file will use them.
\ifx\du\undefined
  \newlength{\du}
\fi
\setlength{\du}{15\unitlength}
\begin{tikzpicture}
\pgftransformxscale{1.000000}
\pgftransformyscale{-1.000000}
\definecolor{dialinecolor}{rgb}{0.000000, 0.000000, 0.000000}
\pgfsetstrokecolor{dialinecolor}
\definecolor{dialinecolor}{rgb}{1.000000, 1.000000, 1.000000}
\pgfsetfillcolor{dialinecolor}
\definecolor{dialinecolor}{rgb}{1.000000, 1.000000, 1.000000}
\pgfsetfillcolor{dialinecolor}
\pgfpathellipse{\pgfpoint{31.000000\du}{12.000000\du}}{\pgfpoint{2.000000\du}{0\du}}{\pgfpoint{0\du}{2.000000\du}}
\pgfusepath{fill}
\pgfsetlinewidth{0.100000\du}
\pgfsetdash{}{0pt}
\pgfsetdash{}{0pt}
\pgfsetmiterjoin
\definecolor{dialinecolor}{rgb}{0.000000, 0.000000, 0.000000}
\pgfsetstrokecolor{dialinecolor}
\pgfpathellipse{\pgfpoint{31.000000\du}{12.000000\du}}{\pgfpoint{2.000000\du}{0\du}}{\pgfpoint{0\du}{2.000000\du}}
\pgfusepath{stroke}
% setfont left to latex
\definecolor{dialinecolor}{rgb}{0.000000, 0.000000, 0.000000}
\pgfsetstrokecolor{dialinecolor}
\node at (31.000000\du,12.195000\du){CLIENT};
\definecolor{dialinecolor}{rgb}{1.000000, 1.000000, 1.000000}
\pgfsetfillcolor{dialinecolor}
\pgfpathellipse{\pgfpoint{40.051131\du}{12.044456\du}}{\pgfpoint{2.051131\du}{0\du}}{\pgfpoint{0\du}{2.044456\du}}
\pgfusepath{fill}
\pgfsetlinewidth{0.100000\du}
\pgfsetdash{}{0pt}
\pgfsetdash{}{0pt}
\pgfsetmiterjoin
\definecolor{dialinecolor}{rgb}{0.000000, 0.000000, 0.000000}
\pgfsetstrokecolor{dialinecolor}
\pgfpathellipse{\pgfpoint{40.051131\du}{12.044456\du}}{\pgfpoint{2.051131\du}{0\du}}{\pgfpoint{0\du}{2.044456\du}}
\pgfusepath{stroke}
% setfont left to latex
\definecolor{dialinecolor}{rgb}{0.000000, 0.000000, 0.000000}
\pgfsetstrokecolor{dialinecolor}
\node at (40.051131\du,12.239456\du){SERVER};
\pgfsetlinewidth{0.100000\du}
\pgfsetdash{}{0pt}
\pgfsetdash{}{0pt}
\pgfsetbuttcap
{
\definecolor{dialinecolor}{rgb}{0.000000, 0.000000, 0.000000}
\pgfsetfillcolor{dialinecolor}
% was here!!!
\pgfsetarrowsend{to}
\definecolor{dialinecolor}{rgb}{0.000000, 0.000000, 0.000000}
\pgfsetstrokecolor{dialinecolor}
\draw (31.000000\du,14.000000\du)--(31.000000\du,24.000000\du);
}
\pgfsetlinewidth{0.100000\du}
\pgfsetdash{}{0pt}
\pgfsetdash{}{0pt}
\pgfsetbuttcap
{
\definecolor{dialinecolor}{rgb}{0.000000, 0.000000, 0.000000}
\pgfsetfillcolor{dialinecolor}
% was here!!!
\pgfsetarrowsend{to}
\definecolor{dialinecolor}{rgb}{0.000000, 0.000000, 0.000000}
\pgfsetstrokecolor{dialinecolor}
\draw (40.051100\du,14.088900\du)--(40.000000\du,24.000000\du);
}
\pgfsetlinewidth{0.100000\du}
\pgfsetdash{}{0pt}
\pgfsetdash{}{0pt}
\pgfsetbuttcap
{
\definecolor{dialinecolor}{rgb}{0.000000, 0.000000, 0.000000}
\pgfsetfillcolor{dialinecolor}
% was here!!!
\pgfsetarrowsend{to}
\definecolor{dialinecolor}{rgb}{0.000000, 0.000000, 0.000000}
\pgfsetstrokecolor{dialinecolor}
\draw (31.000000\du,16.000000\du)--(40.030660\du,18.053340\du);
}
\pgfsetlinewidth{0.100000\du}
\pgfsetdash{}{0pt}
\pgfsetdash{}{0pt}
\pgfsetbuttcap
{
\definecolor{dialinecolor}{rgb}{0.000000, 0.000000, 0.000000}
\pgfsetfillcolor{dialinecolor}
% was here!!!
\pgfsetarrowsend{to}
\definecolor{dialinecolor}{rgb}{0.000000, 0.000000, 0.000000}
\pgfsetstrokecolor{dialinecolor}
\draw (40.020440\du,20.035560\du)--(31.000000\du,22.000000\du);
}
% setfont left to latex
\definecolor{dialinecolor}{rgb}{0.000000, 0.000000, 0.000000}
\pgfsetstrokecolor{dialinecolor}
\node[anchor=west] at (35.510220\du,21.017780\du){};
% setfont left to latex
\definecolor{dialinecolor}{rgb}{0.000000, 0.000000, 0.000000}
\pgfsetstrokecolor{dialinecolor}
\node[anchor=west] at (32.000000\du,20.000000\du){};
\definecolor{dialinecolor}{rgb}{1.000000, 1.000000, 1.000000}
\pgfsetfillcolor{dialinecolor}
\fill (33.000000\du,16.000000\du)--(33.000000\du,17.900000\du)--(38.057500\du,17.900000\du)--(38.057500\du,16.000000\du)--cycle;
\pgfsetlinewidth{0.100000\du}
\pgfsetdash{}{0pt}
\pgfsetdash{}{0pt}
\pgfsetmiterjoin
\definecolor{dialinecolor}{rgb}{0.000000, 0.000000, 0.000000}
\pgfsetstrokecolor{dialinecolor}
\draw (33.000000\du,16.000000\du)--(33.000000\du,17.900000\du)--(38.057500\du,17.900000\du)--(38.057500\du,16.000000\du)--cycle;
% setfont left to latex
\definecolor{dialinecolor}{rgb}{0.000000, 0.000000, 0.000000}
\pgfsetstrokecolor{dialinecolor}
\node at (35.528750\du,17.145000\du){"ESPRESSO"};
\definecolor{dialinecolor}{rgb}{1.000000, 1.000000, 1.000000}
\pgfsetfillcolor{dialinecolor}
\fill (33.000000\du,20.000000\du)--(33.000000\du,21.900000\du)--(38.075000\du,21.900000\du)--(38.075000\du,20.000000\du)--cycle;
\pgfsetlinewidth{0.100000\du}
\pgfsetdash{}{0pt}
\pgfsetdash{}{0pt}
\pgfsetmiterjoin
\definecolor{dialinecolor}{rgb}{0.000000, 0.000000, 0.000000}
\pgfsetstrokecolor{dialinecolor}
\draw (33.000000\du,20.000000\du)--(33.000000\du,21.900000\du)--(38.075000\du,21.900000\du)--(38.075000\du,20.000000\du)--cycle;
% setfont left to latex
\definecolor{dialinecolor}{rgb}{0.000000, 0.000000, 0.000000}
\pgfsetstrokecolor{dialinecolor}
\node at (35.537500\du,21.145000\du){"ESPRESSO"};
\end{tikzpicture}
}
\caption{Example: A \textbf{Leaky Coffee} session without $FlavourRequest$ and $FlavourResponse$}
\label{Fig: Leaky-Coffee Example2}
\end{figure}
\end{example}

\subsection{Analysis}

Similar to most cryptography researches, we assume the implementation of \textbf{Leaky Coffee} are made public. We model our adversary to be given the full knowledge that is observable through a sniffer\footnote{In this project, we used Wireshark\cite{Wireshark}.}, as those displayed in \Cref{Fig: Captured Leaky Coffee packets}.

\begin{figure}[H] 
\centering
\resizebox{14cm}{!}
{\includegraphics{./Pics/Wireshark01.png}}
\caption{Captured Leaky Coffee packets}
\label{Fig: Captured Leaky Coffee packets}
\end{figure}

\subsection{Session Detection and Segmentation}
The existence of packets implies that a session is taking place between CLIENT and SERVER.

Further more, since there is a significant difference in the timestamp intervals between continuous packets from same session and different session, one can group packets by their timestamps. Typically a threshold of $4$ seconds is good enough for \textbf{Leaky Coffee}. As we can see in the “Time” column in \Cref{Fig: Captured Leaky Coffee packets}\footnote{The in-continuous packet number is a result for filtering DTLS packets on the hosting machine.},
\begin{itemize}
\item Packet $70$ to $73$ is a session with $FlavourRequest$ and $FlavourResponse$.
\item Packet $82$ to $85$ is another session with $FlavourRequest$ and $FlavourResponse$.
\item Packet $86$ and $87$ is a session without $FlavourRequest$ and $FlavourResponse$.
\item Packet $96$ and $97$ is another session without $FlavourRequest$ and $FlavourResponse$.
\item ...
\end{itemize}

Once a session has been segmented, we can immediately label them as described in \Cref{Def: Leaky Coffee Session}.

However,  timing information could be strongly affected by the environment; therefore the time difference threshold for a real sensor network packets could be different, even hard to define.

\subsection{Plaintext Guessing} \label{Sec: Plaintext Guessing}
Similar to \textbf{Odd or Even}(\Cref{Sec: Odd or Even}), ciphertexts exchanged in \textbf{Leaky Coffee} has distinguish length which could possibly be exploited to recover the plaintexts.

To formalise it, we model the leakage through ciphertext lengths as a channel\cite{Information_Theory}, inspired by \cite{Web2}. The general idea is to view the ciphtertext lengths as an input and plaintext the output, then the leakage problem is immediately equivalent  to the decoding problem of such a channel. 

In order to do this, we first construct a  \textbf{Forward Channel} that “encodes” plaintexts to ciphertext lengths. The \textbf{Leakage Channel}, which “decodes” ciphertext lengths to plaintexts, is therefore followed by as the reversion of \textbf{Forward Channel}.

\begin{definition} \label{Def: Channels}
Let $\mathbb{X}$ be the set of possible plaintexts and $\mathbb{Y}$ the set of possible packet lengths in bytes. The \textbf{Forward Channel} is then given as  $\bf{W(y|x)}$ and \textbf{Leakage Channel} as $\bf{W^{-1}(x|y)}$ where both $x \in \mathbb{X}$ and $y \in \mathbb{Y}$.
\end{definition}

\Cref{Ex: Single_Order} describes an example of how Forward Channel and Leakage Channel are constructed.

\begin{example} \label{Ex: Single_Order}
Take $Order$ for example. According to \Cref{Def: Order}, we have:
\begin{equation*}
\mathbb{X} = \{\text{“AMECINANO”}, \text{“CAPPUCINO”}, \text{“MOCHA”}, \text{“ESPRESSO”}\}
\end{equation*}

Hence the Forward Channel is as \Cref{Tbl: Forward Channel for Order}. 
\begin{table}[H]
\begin{center}
{\begin{tabular}{c|l|l|l||l|}
$W(y|x)$          & 5 & 8 & 9 & $P$\\
\hline
"AMERICANO" &   &  & 1  & $1/4$ \\
\hline
"CAPPUCINO" &   &   & 1  & $1/4$\\
\hline
"MOCHA"     & 1 &   &    & $1/4$\\
\hline
"ESPRESSO"  &   &  1  &   & $1/4$\\
\hline
\end{tabular}
}
\end{center}
\caption{Forward Channel for $Order$}
\label{Tbl: Forward Channel for Order}
\end{table}

The ciphertext length is fixed once the plaintext is given; therefore the probability of a ciphertext length could either be $0$ or $1$.

The joint probability immediately follows by:
\begin{equation}
(\widehat{W}P)(x,y) = P(x)W(y|x)
\end{equation}

Which results into \Cref{Tbl: Joint Probability of (X Y) for Order}.

\begin{table}[H]
\begin{center}
{\begin{tabular}{c|c}

$\widehat{W}P$  & P     \\ \hline
("AMERICANO",9) & $1/4$ \\ \hline
("CAPPUCINO",9) & $1/4$ \\ \hline
("MOCHA",5)     & $1/4$ \\ \hline
("ESPRESSO",8)  & $1/4$ \\ \hline
\end{tabular}
}
\end{center}
\caption{Joint Probability of $(\mathbb{X},\mathbb{Y})$ for $Order$}
\label{Tbl: Joint Probability of (X Y) for Order}
\end{table}

The marginal probability $P(Y=y)$ is obtained by applying the law of total probability:
\begin{equation}
P(Y=y) = \sum\limits_{x \in \mathbb{X}}{\widehat{W}P(x,y)}
\end{equation}

The result is shown in \Cref{Tbl: Marginal Probabilities of y for Order}.

\begin{table}[H]
\begin{center}
{\begin{tabular}{|c|c|}
\hline
 $y$ & P     \\ \hline
5 & $1/4$ \\ \hline
8 & $1/4$ \\ \hline
9 & $1/2$ \\ \hline
\end{tabular}}
\end{center}
\caption{Marginal probabilities of $y$ for $Order$}
\label{Tbl: Marginal Probabilities of y for Order}
\end{table}

Finally we can construct the Leakage Channel using Bayes’ theorem:
\begin{equation}
P(x|y) = {\frac {P(x)P(y|x)} {P(y)}} 
\end{equation}

The result is shown as \Cref{Tbl: Leakage Channel for Order}.

\begin{table}[H]
\begin{center}
{\begin{tabular}{c|c|c|c|c|}
$W^{-1}(x|y)$  & “AMERICANO” & “CAPPUCINO” & “ESPRESSO” & “MOCHA” \\ \hline
5 &             &             &            & $1$     \\ \hline
8 &             &             & $1$        &         \\ \hline
9 & $1/2$       & $1/2$       &            &         \\ \hline
\end{tabular}
}
\end{center}
\caption{Leakage Channel for $Order$}
\label{Tbl: Leakage Channel for Order}
\end{table}

The Leakage Channel(\Cref{Tbl: Leakage Channel for Order}) can be then used as a guideline for an eavesdropper adversary to recover the plaintext being transmitted.

In fact as an implementation optimisation of this method, one can directly construct the Leakage Channel by iterating through each column of the Forward Channel without storing any whole of the intermediate tables, as described in \Cref{Alg: FC2LC}.

\begin{algorithm}[H]
 \KwIn{\\
 Marginal probabilities $P(X=x)$\;
 Forward Channel $W(y|x)$, where $W_{ij}=P(Y=y_j|X=x_i)$, $i \in |\mathbb{X}|, y \in |\mathbb{Y}|$
 }
 \KwOut{\\
 Leakage Channel $W^{-1}(x|y)$ where $W^{-1}_{ji} = P(X=x_i | Y=y_j)$, $i \in |\mathbb{X}|, y \in |\mathbb{Y}|$
 }
 \Begin{
 \tcp{Iterate over each column of $W(y|x)$}
 \For{$j = 0$; $j < |\mathbb{Y}|$; j++}
 {
  	 \tcp{Reset $J[i]$, the joint probability of $P(x=x_i, y=y_j)$}
 	 $J[i]$ = $\vec{0}$\;
 	\tcp{Compute $P_y$, the marginal probability of $P(Y=y_j)$, by $J[i]$.}
 	\For{$P_y = 0, i=0$; $i < |\mathbb{X}|$; $i++$}
 	{
 		$J_y[i]$ =  $W_{ij}$ * $P(X = x_i)$\;
 		$P_y$ += $J_y[i]$\;
 	} 	
 	\tcp{Compute $W^{-1}(x | y)$}
 	\For{$i=0$; $i < |\mathbb{X}|$; $i++$}
 	{
 		$W^{-1}_{ji}$ = $J[i]$/ $P_y$\;
 	}
 }
return $W^{-1}(x|y)$\;
}
 \caption{FC2LC} \label{Alg: FC2LC}
\end{algorithm}

\end{example}

One underlining problem in this method is the need to enumerate the plaintext. In some scenarios, it is possible to mitigate this problem by reducing the size of $\mathbb{X}$ by wrapping plaintexts. The cost of such mitigation is the loss of resolution in the recovered plaintext, as shown in \Cref{Ex: Single-OrderFlavour}.

\begin{example} \label{Ex: Single-OrderFlavour}
$Coffee$(\Cref{Def: Coffee}) is relatively harder to enumerate as the various combination of $Sugar$ and $Milk$. However, if the adversary aims only to guess the first part, $Order$, then the construction of Leakage Channel can be done much more efficiently by giving up the resolution of distinguishing $Sugar$ and $Milk$. To be more specifically, the construction of Leakage Channel could be simplified by wrapping the “less important” part, in this example $Sugar$ and $Milk$, of plaintexts.

Since we know the probabilities for $Sugar$ and $Milk$, therefore we can construct Forward Channel as \Cref{Tbl: Forward Channel for Coffee}.

\begin{table}[H]
\centering
\resizebox{\textwidth}{!}{
{\begin{tabular}{c|c|c|c|c|c|c|c|c|c|c|c||c|}
$W(y|x)$    & 5    & 6   & 7    & 8   & 9    & 10  & 11   & 12  & 13   & 14  & 15   & $P(X=x)$ \\ \hline
“AMERICANO” &      &     &      &     & 1/16 & 1/8 & 3/16 & 1/4 & 3/16 & 1/8 & 1/16 & 1/4      \\ \hline
“CAPPUCINO” &      &     &      &     & 1/16 & 1/8 & 3/16 & 1/4 & 3/16 & 1/8 & 1/16 & 1/4      \\ \hline
“MOCHA”     & 1/16 & 1/8 & 3/16 & 1/4 & 3/16 & 1/8 & 1/16 &     &      &     &      & 1/4      \\ \hline
“ESPRESSO”  &      &     &      & 1   &      &     &      &     &      &     &      & 1/4      \\ \hline
\end{tabular}
}
}
\caption{Forward Channel for $Coffee$}
\label{Tbl: Forward Channel for Coffee}
\end{table}

Then the Leakage Channel for $Coffee$ can be obtained by \Cref{Alg: FC2LC}, as shown in \Cref{Tbl: Leakage Channel for Coffee}.
\begin{table}[H]
\centering
%\resizebox{\textwidth}{!}{
{\begin{tabular}{c|c|c|c|c|}
$W^{-1}(x|y)$  & “AMERICANO” & “CAPPUCINO” & “ESPRESSO” & “MOCHA” \\ \hline
5  &           &           &          & 1     \\ \hline
6  &           &           &          & 1     \\ \hline
7  &           &           &          & 1     \\ \hline
8  &           &           & 4/5      & 1/5   \\ \hline
9  & 1/5       & 1/5       & 3/5      &       \\ \hline
10 & 1/3       & 1/3       & 1/3      &       \\ \hline
11 & 3/7       & 3/7       & 1/7      &       \\ \hline
12 & 1/2       & 1/2       &          &       \\ \hline
13 & 1/2       & 1/2       &          &       \\ \hline
14 & 1/2       & 1/2       &          &       \\ \hline
15 & 1/2       & 1/2       &          &       \\ \hline
\end{tabular}
}
%}
\caption{Leakage Channel for $Coffee$}
\label{Tbl: Leakage Channel for Coffee}
\end{table}
\end{example}

To summarise, a Leakage Channel can be constructed by the following steps:
\begin{description}
\item[1] Wrap the “unimportant part”.(optional)
\item[2] Construct the Forward Channel.
\item[3] Construct the Leakage Channel by \Cref{Alg: FC2LC}.
\end{description}

The length analysis can be further improved by jointly analyse more packets, such as applying them to Hidden Markov Model. For example, if two packets of length $8$ continuously appears in a \textbf{Leaky Coffee} session then we can immediately conclude that the plaintexts are “ESPRESSO” and “ESPRESSO” as an “ESPRESSO” $Order$ will definitely results into an “ESPRESSO” $Coffee$. However, this method has a strong dependency on the application and thus we will not discuss further in this paper.

\subsection{Estimate plaintext distribution}
In \Cref{Sec: Plaintext Guessing} we have demonstrated how to construct a Leakage Channel given the pre-knowledge of plaintext and their probabilities. It is also possible to do it the other way around, that is, to estimate the distribution of plaintexts through ciphertexts.

The general idea is that the distribution of length (as presented as $P$ column in \Cref{Tbl: Order3}) can actually be observed from the ciphertext; therefore we can revert the process and use it to estimate the distribution of plaintext.

\begin{example}
Assume we have a sample of encrypted $Order$ packet collected where we estimated the its length distribution as following: 
\begin{table}[H]
\begin{center}
{\begin{tabular}{|c|c|}
\hline
 $y$ & P     \\ \hline
5 & $d_1$ \\ \hline
8 & $d_2$ \\ \hline
9 & $d_3$ \\ \hline
\end{tabular}
}
\end{center}
\caption{Estimated length distribution from encrypted $Order$ packets}
\label{Tbl: Estimated length distribution from encrypted $Order$ packets}
\end{table}

Similar to \Cref{Exmp: Single-Order}, the first step is to construct a Content-Length channel. The difference is that this time we do not have the pre-knowledge of plaintext distribution; therefore we represent the content distribution as unknown variables $p_i$ for each content.

\begin{table}[H]
\begin{center}
{\begin{tabular}{c|l|l|l||l|}
$W(y|x)$          & 5 & 8 & 9 & $P$\\
\hline
"AMERICANO" &   &  & 1  & $p_1$ \\
\hline
"CAPPUCINO" &   &   & 1  & $p_2$\\
\hline
"MOCHA"     & 1 &   &    & $p_3$\\
\hline
"ESPRESSO"  &   &  1  &   & $p_4$\\
\hline
\end{tabular}
}
\end{center}
\caption{Content-Length Channel with unknown distibution of $Order$}
\label{Tbl: Content-Length Channel with unknown distibution of $Order$}
\end{table}


%Since we are only interested in $p_1$ and $p_3$ and the appearance of each content is exclusive, \Cref{Tbl: Content-Length Channel with unknown distibution of $Order$} can be rewritten as \Cref{Tbl: Rewritten Content-Length Channel with unknown distibution of $Order$}.
%
%\begin{table}[H]
%\begin{center}
%{\begin{tabular}{c|c|c|c||c|}
$W(y|x)$          & 5 & 8 & 9 & $P$\\
\hline
"AMERICANO" &   &  & 1  & $p_1$ \\
\hline
"MOCHA"     & 1 &   &    & $p_3$\\
\hline
"ESPRESSO" or "CAPPUCINO"  &   &  $p_4 / (p_2 + p_4)$  &  $p_2 / (p_2 + p_4)$  & $p_4 + p_2$\\
\hline
\end{tabular}
}
%\end{center}
%\caption{Rewritten Content-Length Channel with unknown distibution of $Order$}
%\label{Tbl: Rewritten Content-Length Channel with unknown distibution of $Order$}
%\end{table}

Their joint distribution follows immediately.

\begin{table}[H]
\begin{center}
{\begin{tabular}{c|c}

$\widehat{W}P$  & P     \\ \hline
("AMERICANO",9) & $p_1$ \\ \hline
("CAPPUCINO",9) & $p_2$ \\ \hline
("MOCHA",5)     & $p_3$ \\ \hline
("ESPRESSO",8)  & $p_4$ \\ \hline
\end{tabular}
}
\end{center}
\caption{Joint distribution of $(Order, l)$ with unknown distribution of $Order$}
\label{Tbl: Joint distribution of $(Order, l)$ with unknown distribution of $Order$}
\end{table}

Then we can compute the marginal distribution of length.

\begin{table}[H]
\begin{center}
{\begin{tabular}{|c|c|}
\hline
 $y$ & P     \\ \hline
5 & $p_3$ \\ \hline
8 & $p_4$ \\ \hline
9 & $p_1 + p_2$ \\ \hline
\end{tabular}}
\end{center}
\caption{Marginal distribution of $l$ with unknown distribution of $Order$}
\label{Tbl: Marginal distribution of $l$ with unknown distribution of $Order$}
\end{table}

By linking \Cref{Tbl: Marginal distribution of $l$ with unknown distribution of $Order$} with \Cref{Tbl: Estimated length distribution from encrypted $Order$ packets} we have the following constrains to the content distribution:

\begin{equation} \label{Eq: Plaintext Distribution Estimation}
\left \{
\begin{aligned}
p_3 = d_1\\
p_4 = d_2\\
p_1 + p_2 = d_3\\
\sum\limits_{i = 1}^4{p_i} = 1
\end{aligned}
\right
\end{equation}

Any solution satisfies \Cref{Eq: Plaintext Distribution Estimation} can be viewed as a reasonable guess to the distribution of the contents.

\end{example}
