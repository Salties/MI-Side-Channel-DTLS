\chapter{Plan}

The short term plan is to finish the toy application analysis in \Cref{Chp: Progress To Date}. There are these directions in this task:
\begin{enumerate}
\item So far we have mainly focused on demonstrating the phenomenon similar to those described in \cite{Web1} and \cite{Web2} exists in our set up, but there could possibly be more. For example, it is probably worth study whether it is feasible to deduce the plaintext distribution in \Cref{Sec: Leaky Coffee} given encrypted packets.
\item To see if the attacks in \Cref{Sec: Leaky Coffee} can be optimised.
\item Apply and evaluate some countermeasures to these attacks.
\item Only a few, nearly none, attempt has been made to study types of attack other than \cite{Web1} and \cite{Web2}; hence this might also be worth trying.
\end{enumerate}

One of the greatest deficiency in the work so far is the risk of being alienated from reality since the toy applications are designed in an extremely abstracted way whilst the real world application could be much more complicated and will not be intentionally designed to be vulnerable to attacks. Therefore as a long term plan, it is very important to get in touch with some actual application and evaluate those we have developed from the toys. Having said so, one of the greatest challenges is  the immaturity of IoT; hence following the trend of related technologies would be an important and constant task in this project.