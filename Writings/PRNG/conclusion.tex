\section{Conclusion}
In this paper we presented a study on the RNG design of CC2538. First, we revised the problem that using a 16 bit LFSR as PRNG is a bad idea and demonstrated how this design flaw can be exploited to break DTLS running on these devices. Secondly we presented a study to its seeding method and showed how such it could be remotely tampered by an adversary sending jamming signal to the device.

In fact the same RNG design has also been adopted many other products in the CC series including CC2420\cite{CC2420Manual}, CC2430\cite{CC2430Manual}, CC2520\cite{CC2520Manual} and CC253X, CC2540/41 series\cite{CC2530Manual}. We imagine all these products suffer the same problems. Fortunately the latest CC26XX/CC13XX\cite{CC26XXManual} has abandoned this design and implemented a dedicated RNG which TI describes as: (Chapter 16 in CC26XX/CC13XX Manual\cite{CC26XXManual})
\begin{quote}
The true random number generator (TRNG) module provides a true, nondeterministic noise source for the
purpose of generating keys, initialization vectors (IVs), and other random number requirements. The
TRNG is built on 24 ring oscillators that create unpredictable output to feed a complex nonlinear
combinatorial circuit. That post-processing of the output data is required to obtain cryptographically secure
random data.
\end{quote}

We sincerely hope this TRNG will provide the future IoT applications a secure RNG.

\section{Acknowledgement}
We have many thanks to (alphabetically) George Oikonomou for providing us much help in Contiki OS and the OpenMote devices, Geoff Hilton who helped us on RF designs and Jake Longo Galea who offered many signal processing advises.