\chapter{Potential Leakage Sources}

In this chapter, we discuss the potential leakage sources in the environment we have set up in \Cref{Chp: Experiment Setup}. 

%\section{A review of review: what could be leaked?} \label{Sec: A Review of Review}
%What is ``Information''

%The first and fundamental question that must be clarified in this project is: what is ``information''?
%
%If we review those attacks described in \Cref{Chp: Literature Review}:
%
%\begin{itemize}
%	\item In literatures about the flaws in implementations of algorithms and protocols, ``information'' is the plaintext, usually represented as a numerical value, such as \cite{802154sec} \cite{rfc7457} \cite{CompressionRatioAttack} \cite{PaddingOracle}. Or it could be the secret key materials in many cryptographic side channel attack literatures, such as \cite{DPA}.
%	\item In Traffic Analysis Attacks against web applications, ``informaiton'' is the user input, such as \cite{PinpointWeb} \cite{SearchAttack}, or the content of websites such as \cite{WebSideChannel}.
%	\item When we talk about website fingerprinting such as \cite{WebsiteFingerprint} \cite{Peekaboo} and \cite{PClassifier}, ``information'' is the identity of websites. 
%	\item The category only gets even more messy and trivial when we look at other Traffic Analysis Attacks, such as the spoken language and phrases in \cite{VoIPLanguage} and \cite{VoIPPhrases}, user event and OS of Apple product in \cite{AppleMessage} or even the ambient change and motion event in \cite{Video}.
%\end{itemize}
%
%``Information'' is always a relative concept to ``application'', and so is ``leakage''. 

\section{Objects} \label{Sec: Objects}

Without loss of generality, In this report, we are interested on leakages with respect to the following objects:

\begin{description}[style=nextline]
	\item[Content and Size of Application Data]
	The implication of Application Data varies is specified in each application.
	
	\item[Cryptographic Key]
	This represents the corresponding key materials in the security measure.
	
	\item[Network Topology]
	This implies how Sensor Nodes are connected to each other.
	
	%\item[Application Specific Feature]
\end{description}

In this chapter we give a theoretical general analysis of the information leakage from the Traffic Analysis aspect. The inspection of protocol and implementation flaws are done in later chapters for each security measure.

Notice that Traffic Analysis is highly application specific. In this report, we discuss only the applications described in \Cref{Sec: Applications}. 

\section{Observables} \label{Sec: Observables}

As we have explained in \Cref{Chp: Experiment Setup}, we assume the adversary has full access to all traffic being transmitted in the WSN. Generally speaking, observables in captured traffic can be classified into two categories:
\begin{description}[style=nextline]
	\item[Implicit observables]
	 These imply the data an adversary can access without inspecting the content of packets.
	\item[Explicit observables]
	 These imply the data given in the unencrypted part of packets.
\end{description}

In this research, we do not attempt to break the cryptographic primitives in the security measures; therefore we assume the value of ciphertext is independent to the plaintext and key. As a result, we do not consider the encrypted data as observables in this report.

\subsection{Implicit Observables} \label{Subsec: Implicit Observables}

In this report, we consider the following implicit observables in captured WSN packets:

\begin{itemize}
	\item Packet Size
	\item Packet Timing
	\item Number of Packets
\end{itemize}

These packet feature can be further used to compute other packet features, such as bandwidth. We define these computable features as part of implicit observables.

Above them, there exist other implicit observables we do not concern in this report, including:
\begin{description}[style=nextline]
	\item[Presence of Packets]
	Apparently, the presence of a packet implies the existence of a Sensor Node device, which is kind of an information leakage. However, we assume the adversary has prior knowledge of the existence of WSN and do not take this into account.
	
	\item[RSSI]
	Receive Signal Strength Indicator, RSSI, indicates the RF signal strength from a Sensor Node. Practically speaking, by measuring the RSSI of frames sent from a specific source MAC address, the adversary is capable to reveal the physical location of a Sensor Node. In this report we do not consider RSSI as its physical character is beyond the scope of this project. Instead, we assume the adversary has prior knowledge of the geographic information of the Sensor Nodes.
\end{description}

The implicit observables are accessible in all scenarios in our experiments based on the assumption that the adversary has full access to all packets in the WSN.

\subsection{Explicit Observables}

We have explained in \Cref{Chp: Experiment Setup} that there are two security measures implemented on our platform, namely noncoresec and DTLS. Referring to \Cref{Chp: Building Blocks}, these measures are implemented at different layers. The observables are therefore different for noncoresec and DTLS. 

\subsubsection{noncoresec} \label{Subsubsec: Explicit noncoresec}

As explained in \Cref{Subsec: 802.15.4 Security Implementation in Contiki}, noncoresec is the 802.15.4 Security implementation on Contiki. When noncoresec is enabled, all captured frames are in the form we explained in \Cref{Subsec: 802.15.4 MAC} and \Cref{Subsec: 802154 Sec}. The 802.15.4 MAC Header is the only observable part of a packet.

We stated in \Cref{Subsec: noncoresec in experiment} that we always use the highest Security Level of 802.15.4 Security; therefore:
\begin{itemize}
	\item MAC Payload is always authenticated and encrypted in AES-128 with CCM* mode.
	\item Security Level is constantly 0x7.
	\item Key Strategy is constantly 0x0 as noncoresec does not implement any key management.
\end{itemize}

As a result of MAC Payload being authenticated and encrypted, 
\begin{itemize}
	\item The adversary cannot join the 6LoWPAN network as we have explained in \Cref{Subsec: noncoresec in experiment}.
	\item The adversary cannot distinguish an IPv6 packet and an ICMPv6 packet by MAC Payload as it is encrypted. 
\end{itemize}

The IPv6 packets effectively are the packets with Application Data in our experiments. ICMPv6 packets on the other hand are mostly RPL messages.

\subsubsection{DTLS} \label{Subsubsec: Explicit DTLS}

As explained in \Cref{Chp: Building Blocks}, when DTLS is used, the explicit observables are: 

\begin{itemize}
	\item 802.15.4 MAC Header as explained in \Cref{Subsec: 802.15.4 MAC}.
	
	\item Compressed IPv6 Header as explained in \Cref{Subsec: 6LoWPAN Adaptation Sub Layer}, \Cref{Subsec: IPv6 Data Packets} and \Cref{Subsec: ICMPv6}.
	
	\item UDP Header as explained in \Cref{Subsec: UDP}.
	
	\item DTLS Header as explained in \Cref{Subsec: DTLS}.
\end{itemize}

Since DTLS is built on Application Layer, therefore it does not impose any setting to the lower layer headers. 

\subsection{Traces} \label{Subsec: Traces}

Before discussing the ``traces'' in WSN applications, as a comparison, we review the different definition of ``traces'' in the literatures we reviewed in \Cref{Chp: Literature Review}:

\begin{itemize}
	\item In literatures about Traffic Analysis Attacks against web applications\cite{WebSideChannel}\cite{PinpointWeb}\cite{SearchAttack}, a ``trace'' refers to the continuous\footnote{We would argue that the term ``continuous'' is ambiguous; nevertheless it is the best definition we can thought of.} packets triggered by an user input.
	\item In Website Fingerprinting literatures\cite{WebsiteFingerprint} \cite{HClassifier} \cite{PClassifier} \cite{Peekaboo}, a ``trace'' refers to the continuous packets that triggered by a browser requesting a website.
	\item Other attacks use more specific application dependent definitions of a ``trace''. \cite{VoIPLanguage} and \cite{VoIPPhrases} defined a ``trace'' to be the continuous packets during a VoIP conversation. \cite{Video} defines it to be the continuous packets during a video conversation. \cite{AppleMessage} does not even use the term ``trace'' at all\footnote{Actually it did used once as a verb equivalent to ``track''.} as it only analyses a single packet.
\end{itemize}

Practically, the ``trace'' of an Internet application can be defined by filtering packets belong to the same TCP connection,or UDP packets with the same entities\footnote{An entity is the combination of an IP address and a port}. However, we cannot apply the same method on many WSN applications due to the change of protocol suite and application nature. For example, TCP is not used as explained in \Cref{Chp: Building Blocks} and communication are mostly done with ephemeral ports. Further more, the WSN applications perform Machine To Machine, M2M, communication which has much less external intervention, makes it harder to group packets into ``traces''.

Further more, due to constrained resources, WSN application designs tend to be simple and stateless. That is to say the computation between Sensor Nodes must be simple enough to be performed on such devices and data transmission is minimised. From this aspect, we argue that the simple applications we described in \Cref{Sec: Applications} have captured these most significant characteristics of WSN applications.

In order to carry on the analysis, we explicitly define a model for the WSN applications.

\begin{definition} \label{Def: Session}
A Session is defined as a series of events where a Sensor Node reports Application Data to a Manager. A Session includes an optional Request sent from Manager to Sensor Node, and a Response containing the Application Data sent from Sensor Node to Manager.
\end{definition}

\begin{definition} \label{Def: Trace}
A Trace is defined as all captured packets within one Session.
\end{definition}

We assume the adversary is capable to distinguish Sessions. Practically, the time interval of packets within the same Session are significantly less then the interval between Sessions; therefore the Sessions can be isolated by grouping packets by time.

For most of our applications described in \Cref{Sec: Applications}, there are typically two packets in a trace. The first one corresponds to the Request and the second corresponds to Response. Other applications, e.g. the broadcast keyllsec application in \Cref{Sec: Applications} or a CoAP application using OBSERVE option as we explained in \Cref{Subsec: CoAP}, there might be only one Response packet in a trace without a Request.

As of this project we consider only information leakage within one Session. One might argue that there is potential information leakage when jointly analyse traces from different Sessions, such as the frequency of Requests can ``leak'' itself. We do not take these leakage into concern as it does not make much sense without considering within a higher layer application structure.  Also in our experiments, we make a reasonable assumption based on our code that each Session is independent, as each execution of the code is independent to its last execution.

%Further more, in our applications the content of Request is a constant 3 bytes ASCII string ``GET''. We assume this known by the adversary. We make this assumption to focus on the leakage of the Application Data in Responses. 

\section{Theoretical Analysis}

In this section we provide a theoretical analysis over all observables we explained in \Cref{Sec: Observables}. It is done in two approaches:

\begin{enumerate}
	\item Cross reference the intersection of the observables we explained in \Cref{Sec: Observables} with the known exploitable side channel information we introduced in \Cref{Sec: Traffic Analysis Attacks}
	\item Semantically analyse the observables. 
\end{enumerate}

%We discuss only the leakage over the attributes we specified in \Cref{Sec: A Review of Review}.

For some of the observables, we argue that:

\begin{theorem} \label{Te: IR}
Under the leakage definitions in \Cref{Subsec: Information Theory}, if an observable $Y$ is independent to a secret $X$, then there is $0$ leakage of $X$ over $Y$.
\end{theorem}

\begin{proof}
	%Independent random variables does not leak.
	Since $X$ and $Y$ are independent, therefore
	\begin{eqnarray*}
		\begin{aligned}
			P(x,y) &= P(x)P(y) \\
			P(x|y) &= P(x)
		\end{aligned}
	\end{eqnarray*}
	where $x \in X$ and $y \in Y$.

	For Mutual Information and Capacity, we have:
	\begin{eqnarray*}
		\begin{aligned}
			H(X|Y) 
			&= - \sum_{x \in X} \sum_{y \in Y} P(x,y)\log{P(x|y)} \\
			&= - \sum_{x \in X} \sum_{y \in Y} P(x)P(y)\log{P(x)} \\
			&= \sum_{y \in Y} P(y) (- \sum_{x \in X}P(x)\log{P(x)}) \\
			&= \sum_{y \in Y} P(y) H(X) = H(X) \sum_{y \in Y}{P(y)} \\
			&= H(X)
		\end{aligned}
	\end{eqnarray*}
	
	Therefore
	\begin{eqnarray*}
		\begin{aligned}
			I(X;Y) &= H(X) - H(X|Y) = H(X) - H(X) = 0 \\
			C &= \sup_{\forall P(X)} I(X;Y) = \sup_{\forall P(X)} 0 = 0
		\end{aligned}
	\end{eqnarray*}
	
	Similarly for gain function based leakage\cite{GLeakage},
	\begin{eqnarray*}
		\begin{aligned}
			V_{g}(\pi, C) 
			&= \sum_{y \in Y}{\max_{w \in W}\sum_{x \in X}{\pi[x]C[x,y]g(w,x)}} \\
			&= \sum_{y \in Y}{\max_{w \in W}\sum_{x \in X}{\pi[x]P(y|x)g(w,x)}} \\
			&= \sum_{y \in Y}{\max_{w \in W}\sum_{x \in X}{\pi[x]P(y)g(w,x)}} \\
			&= \sum_{y \in Y}p(y){\max_{w \in W}\sum_{x \in X}{\pi[x]g(w,x)}} \\
			&= \max_{w \in W}\sum_{x \in X}{\pi[x]g(w,x)} = V_{g}(\pi)
		\end{aligned}
	\end{eqnarray*}
	
	Therefore
	\begin{equation*}
		H_g(\pi, C) = -\log{V_g(\pi, C)} = -\log{V_g(\pi)} = H_g(\pi)
	\end{equation*}
	
	Hence 
	\begin{eqnarray*}
		\begin{aligned}
			L_g(\pi, C) &= H_g(\pi) - H_g(\pi,C) = H_g(\pi) - H_g(\pi) = 0\\
			ML_g(C) &= \sup_{\pi} L_g(\pi, C) = \sup_{\pi} 0 = 0
		\end{aligned}
	\end{eqnarray*}
\end{proof}

\begin{corollary} \label{Cor: Constant Leakage}
	An observable $Y$ with a constant value has no leakage under the definitions in \Cref{Subsec: Information Theory}.
\end{corollary}

\begin{proof}
	This directly followed by the fact that a constant $y \in Y$ with $P(y) = 1$ is independent to the secret $X$.
\end{proof}

In this report, we assume the adversary has full knowledge of which application the Sensor Nodes are running and is aware of all the identities of Sensor Nodes in the WSN, including the address information of the Manager. The only unknown secret to the adversary are those objects specified in \Cref{Sec: Objects}. 

\subsection{Cross Reference of Observables with Known Attacks} \label{Subsec: Cross Reference}

Despite the differences in the nature applications, we discuss the applicability of the packet features those have been exploited in Traffic Analysis Attacks, as we have summarised in \Cref{Tbl: Classifiers in Traffic Analysis Literatures}.

\begin{description}[style=nextline]
	\item[Direction]
	In our applications, it is predictable that the first packet is a Request from Manager to Sensor Node and the second is Response from Sensor Node to Manager. In a one packet Session there is only one packet from Sensor Node to Manager, or a broadcast packet without direction. Therefore direction is either constant or not applicable in these scenarios.
	
	\item[Length]
	The is effectively the packet size in implicit observables.
	
	\item[Frequency Distribution of Length]
	The same feature can be computed by packet sizes. However, since there are typically only two packets in a trace, the result is $0.5$ for the length of Request packet and $0.5$ for the length of Response packet. In a one packet Session there is only one value in the distribution with probability of $1$. This feature is applicable but with extremely low entropy of $1$ or $0$.
	
	\item[Size, HTML and Number Markers]
	In a two packet Session there is only one direction change in a trace; thus the markers constantly mark the second packet. In an one packet Session this feature is not applicable.
	
	\item[Total Bytes]
	The same feature can be computed through packet sizes.
	
	\item[Percentage Incoming Packets]
	The term ``incoming'' refers to the direction of web server to the browser in its original Web Fingerprint literature. In our experiments we assumed the adversary monitors all packets in the network; thus there is not an explicit definition of ``incoming'' and ``outgoing''. Even though we can similarly define ``incoming'' as from Sensor Node to Manager, this is feature is fixed given an application. This value is constantly $50\%$ for a two packets Session and $100\%$ for a one packet Session.
	
	\item[Number of Packets]
	Since UDP does not segment any application data, the number of packets in a trace is a constant given an application. 
	
	\item[Total Time]
	In a two packets Session this is exactly the interval between Request and Response. In an one packet session this is not applicable.
	
	\item[Total Per-direction Bandwidth]
	Since there is at most only one packet at each direction, this feature is effectively a single packet size divided by total time for each direction.
	
	\item[Traffic Burst]
	Traffic burst is reduced to packet size in our applications as there is at most only one packet each direction.
\end{description}

According to \Cref{Cor: Constant Leakage}, features with constant value are non leakable features. 

Removing the features with constant value or not applicable, the potential leakage sources from the literatures are:
\begin{itemize}
	\item Length
	\item Total Bytes
	\item Total Time (only for two packets Session)
	\item Total Per-direction Bandwidth
\end{itemize}

%Notice that we ignored Traffic Bursts since it is reduced to packet length in our applications as explained above.

\subsection{Semantic Analysis}

In this section, we provide an analysis over the observables based on their semantics we explained in \Cref{Chp: Building Blocks}.

One exception is the address information. Theoretically speaking, any valid address may be used in any packet and thus could be considered independent to the application. However in practice, different data may use different type of transmission, i.e. unicast, multicast or broadcast; thus the family of address could be a potential leakage source to the Application Data. We provide the analysis within this aspect in later chapters.

Some packets can be observed even if no application is running. These packets, mainly RPL messages, are generated by Contiki kernel to maintain the connectivity of the 6LoWPAN network. Since the existence of these packets is independent to the application running over WSN, we consider them as non leakable due to \Cref{Te: IR}. One exception is the DAO message we explained in \Cref{Subsec: ICMPv6} which could be triggered by the first application packet if the routing is not queried before. In this report, we consider the DAO message independent to the application.

As we have explained in \Cref{Chp: Building Blocks}, the 6LoWPAN protocol suite is constructed layer by layer and each layer is conceptionally transparent to other layers. As a result, most of the headers are in theory independent to the application and thus are non leakable due to \Cref{Te: IR}. For these observables, we expect to see them distributed independently among different applications. Specifically, the choice of cryptographic key is expected to be independent to all contents in the headers.

\subsubsection{Implicit Observables}

The implicit observables are indeed covered by the potential leakage sources we analysed in \Cref{Subsec: Cross Reference}. As a matter of fact, these observables turned out to be the most exploitable features in the literatures we have reviewed in \Cref{Sec: Traffic Analysis Attacks}.

%Linear relationship of length
In our experiments, we noticed that in our security measures, namely noncoresec and DTLS, the underlining cryptographic primitives are both AES-128 with CCM mode where no padding is applied. This effectively indicates that the length of plaintext and ciphertext are expected to have a linear relationship with slope ratio $1$:

\begin{equation} \label{Eq: Linear Length}
	l_{C} = l_{P} + b
\end{equation}

where $l_{C}$ and $l_{P}$ are the length in bytes for packets with and without encryption, and $b$ is a predictable constant induced by security measures such as MIC and change of headers. Practically when the security measure increases the packet length to a value exceeding MTU, the packet will be immediately dropped and cannot be observed by an adversary. However, considering the functionality of WSN, we assume the plaintext is always in a reasonable length that does not result into a ciphertext exceeding MTU.

%Leakage with linear relationship
Modelling the leakage of packet length as a channel $C(l_{C},l_{P})$ as in other Information Theoretic approaches we described in \Cref{Subsec: Information Theory}, we have a deterministic channel such that:
\begin{equation}
	C(l_{P}, l_{C}) = P(l_{P} | l_{C}) = 
	\begin{cases}
		1 &\text{if: } l_{C} = l_{P} + b \\
		0 &\text{otherwise}
	\end{cases}
\end{equation}

and

\begin{equation}
	C^{-1}(l_{C}, l_{P}) = P(l_{C} | l_{P}) = 
	\begin{cases}
		1 &\text{if: } l_{C} = l_{P} + b \\
		0 &\text{otherwise}
	\end{cases}
\end{equation}

So

\begin{equation}
	P(l_{P} , l_{C}) = P(l_{P}) P(l_{C} | l_{P}) =
	\begin{cases}
		P(l_{P}) &\text{if: } l_{C} = l_{P} + b \\
		0 &\text{otherwise}
	\end{cases}
\end{equation}

Therefore\footnote{Information Theory defines $0\log{0} = 0$.},
\begin{equation}
	P(l_{P} , l_{C}) \log{P(l_{P} | l_{C})} = 
	\begin{cases}
		P(l_{P})\log{1} = 0 &\text{if: } l_{C} = l_{P} + b \\
		0 \log{0} = 0 &\text{otherwise}
	\end{cases}
\end{equation}

Hence
\begin{equation}
	H(L_{P} | L_{C}) = - \sum_{l_{C} \in L_{C}} \sum_{l_{P} \in L_{P}}P(l_{P} , l_{C}) \log{P(l_{P} | l_{C})} = - \sum_{l_{C} \in L_{C}} \sum_{l_{P} \in L_{P}} 0 = 0
\end{equation}
where $L_{P}$ and $L_{C}$ are the possible length in bytes of encrypted and unencrypted packets.

In this case, the Mutual Information is:
\begin{equation} \label{Eq: MI in length}
	I(L_{P};L_{C}) = H(L_{P}) - H(L_{P} | L_{C} ) = H(L_{P}) - 0 = H(L_{P})
\end{equation}

For the Capacity, according to \Cref{Eq: MI in length}, $I(L_{P};L_{C})$ has its maximum value when $L_{P}$ is uniformly distributed:
\begin{equation} \label{Eq: Cap in length}
	Capacity = \sup_{\forall L_{P}}{I(L_{P};L_{C})} = \sup_{\forall L_{P}}H(L_{P}) = - \sum_{i = 1}^{|L_{P}|}|L_{P}|^{-1}\log{|L_{P}|^{-1}} = \log{|L_{P}|}
\end{equation}

In another word, \Cref{Eq: MI in length} and \Cref{Eq: Cap in length}  imply that averagely all bits of $l_{P}$ is leaked through $l_{C}$.

For the gain function based leakage\cite{GLeakage}, we realised that it would be hard to quantify the leakage without a specific gain function. Therefore instead, we provide an analysis with min-leakage.

In this case, the Posterior Vulnerability is:
\begin{equation}
	\begin{aligned}
		V(\pi_{L_P}, C^{-1}) 
		&= \sum_{l_{C} \in L_{C}} \max_{l_{P} \in L_{P}} \pi_{L_P}[l_P]C^{-1}[l_P,l_C] \\
		&=  \sum_{l_{C} \in L_{C}} P(l_C) \max_{l_{P} \in L_{P}} P(l_P | l_C) \\
	      &= \sum_{l_{C} \in L_{C}} P(l_C) = 1 \\
	\end{aligned}
\end{equation}

Therefore
\begin{equation}
	\begin{aligned}
		H_{\infty}(\pi_{L_{P}}, C^{-1})
		 &= - \log{V(\pi_{L_{P}}, C^{-1})} = - \log1= 0
	\end{aligned}
\end{equation}

Thus the min-leakage is:
\begin{equation}
	\begin{aligned}
	L(\pi_{L_P}, C^{-1}) 
	 &= H_{\infty}(\pi_{L_P}) - H_{\infty}(\pi_{L_{P}}, C^{-1}) \\
	 &= H_{\infty}(\pi_{L_P}) - 0 \\
	 &= H_{\infty}(\pi_{L_P})
	\end{aligned}
\end{equation}

And finally:
\begin{equation}
	ML(C^{-1}) = \sup_{\pi_{L_P}}{L(\pi_{L_P},C^{-1})} =  \sup_{\pi_{L_P}} H_{\infty}(\pi_{L_P}) = \log{|L_P|}
\end{equation}

This result consists with our intuition and the Capacity in \Cref{Eq: Cap in length} that all bits of $l_P$ are leaked through $l_C$.

\subsubsection{Explicit Observables with noncoresec}

As explained in \Cref{Subsubsec: Explicit noncoresec}, the only explicit observables when using noncoresec is the 802.15.4 MAC Header. 

Referring to \Cref{Subsec: 802.15.4 MAC}, there are four types of MAC Frames. However, both Beacon and Command frame are solely used for connection maintenance and are independent to the application running on Sensor Nodes; therefore we consider them as non leakable sources according to \Cref{Te: IR}.

For the Data frame we explained in \Cref{Subsec: 802.15.4 MAC},
\begin{description}[style=nextline]
	\item[Frame Control]
	Among the flags in Frame Control defined by \cite{802154}, we found only two flags potentially related to upper layer application:
		\begin{description}
			\item[Security Enabled]
			When noncoresec is used, this field is constantly $1$. Due to \Cref{Cor: Constant Leakage}, this is not a potential leakage source.
			\item[ACK Required]
			Since MAC ACK is optional and this field can be set by an upper layer application, we suspect this flag as a potential leakage source.
		\end{description}
	\item[Sequence Number]
	Sequence Number is solely managed by MAC Layer and hence likely to be independent to upper layer application.
%	\item[Address Information]
%	As explained above, address information is excluded from the potential leakage sources.
%	\item[Data]
%	Since the encryption is considered to be a random function, we consider the value Data to be independent from Application Data and Cryptographic Key.
	\item[Frame Checksum]
	Since Frame Checksum is computed from the frame itself and therefore contains only redundant information. We argue that Frame Checksum does not induce any additional information leakage to a packet.
\end{description}

For the Auxiliary Security Header we explained in \Cref{Subsec: 802154 Sec},

\begin{description}[style=nextline]
	\item[Security Level]
	As we explained in \Cref{Chp: Experiment Setup}, this value is constantly $7$ in our experiments and is not a potential leakage source due to \Cref{Cor: Constant Leakage}.
	\item[Frame Counter]
	Similar to Sequence Number, this field is likely to be independent to application.
	\item[Key Strategy]
	noncoresec does not utilise this field at all and thus is constantly $0$. It is not a leakage source due to \Cref{Cor: Constant Leakage}.
\end{description}

To summarise, among the explicit observables with noncoresec,

\begin{itemize}
	\item ACK Request flag in Frame Control is a sceptical leakage source to contents in upper layer payload.
	\item Sequence Number and Frame Counter are likely to be independent to upper layer applications.
\end{itemize}

\subsubsection{Explicit Observables with DTLS}

Comparing to noncoresec, when DTLS is used we have additional information in IPv6 Header and UDP Header as explained in \Cref{Subsubsec: Explicit DTLS}.

The 802.15.4  MAC Header with DTLS is mostly the same as with noncoresec, except that:

\begin{itemize}
	\item The Security Enabled in Frame Control is constantly $0$ as 802.15.4 Security is disabled. We conclude that it is not a potential leakage source due to \Cref{Cor: Constant Leakage}.
	\item There is no Auxiliary Security Header; hence Frame Counter is not applicable within DTLS.
\end{itemize}

As we have explained in \Cref{Subsec: 6LoWPAN Adaptation Sub Layer}, the 6LoWPAN Header Compression is lossless; therefore it contains the identical information as of an uncompressed IPv6 Header. Referring to the IPv6 Header we explained in \Cref{Subsec: IPv6 Data Packets}:

\begin{description}[style=nextline]
	\item[Version]
	This is constantly 0x6.
	\item[Traffic Class and Flow Label]
	Theoretically speaking these values can be set by upper layer application and thus are potential leakage sources. However, in the current version of Contiki release-3.0, their use are not supported and the values are constantly $0$.
	\item[Payload Length]
	This value is redundant as it can be computed by the packet size.
	\item[Next Header]
	When DTLS is used, this field is constantly UDP(0x11).
	\item[Hop Limit]
	This field is solely processed at Network Layer and we suspect it is independent to the upper application. However, it depends on routing and therefore network topology, we expect to see the identical values within the same topology setup among different applications.
%	\item[Source and Destination Address]
%	As explained above, the address information is excluded form the potential leakage sources.
\end{description}

%Length of Compressed and Uncompressed Header
One thing to be noticed is that although the IPv6 Header Compression is lossless and thus does not affect the contents in IPv6 Header, there is still a potential side channel leakage that the difference in size may leak information about the content of IPv6 Header when it is not observable, such as when noncoresec is used. However, the analysis above showed that the only variable fields are the address information. Since we assumed the adversary has prior knowledge about the identity and address information of the WSN, we consider the size of compressed header is predictable to the adversary and thus does not consider this as an information leakage. In our experiments, the same address mode is used for all Sensor Nodes in all experiments; thus all packets have used the same compression format, resulting into identical size of Compressed IPv6 Header.

%UDP Header
For UDP Header described in \Cref{Subsec: UDP},

\begin{description}[style=nextline]
	\item[Source and Destination Port]
	The ports are identical for the same application in our experiments. Practically speaking, the ports may leak which application a packet corresponds to; however, we do not consider this as a leakage as we assumed the adversary has prior knowledge of the application.
	\item[Payload Length]
	Similar to Payload Length in IPv6 Header, this value is redundant.
	\item[Checksum]
	Similar to the MAC Checksum, this value is redundant.
\end{description}

%DTLS Header
The UDP Payload with DTLS can be either a Handshake Messages and a Data Records as explained in  \Cref{Subsec: DTLS}. Since the Handshake process is performed solely by the DTLS module, we consider it independent to any Application Data. However,  the Handshake Messages are potentially linked cryptographic key. We provide an analysis of this subject in later chapters.

With respect to DTLS Data Records,

\begin{description}[style=nextline]
	\item[Content Type]
	In Data Record this is constantly $0x17$.
	\item[Protocol Version]
	In the tinydtls implementation, this is constantly \{0xfe, 0xfd\}.
	\item[Epoch]
	Since no renegotiation took place in our applications, this value is constantly $1$ for all Data Records.
	\item[Sequence Number]
	Similar to MAC Sequence Number, we suspect this value to be independent to Application Data.
	\item[Length]
	This value is as redundant as other length fields in upper layer headers.
%	\item[Fragment]
%	This field contains the ciphertext of Application Data. We assumed its value to be independent to the plaintext and cryptographic key.
\end{description}

In conclusion, for the explicit observables within DTLS,

\begin{itemize}
	\item ACK Request flag is a potential leakage source as with noncoresec.
	\item Hop Limit is theoretically depended on routing and network topology and is likely to be independent to the application.
	\item Sequence Numbers in each layer are likely to be independent to the upper layer application.
\end{itemize}

\subsubsection{Leakage of Network Topology}

As explained in \Cref{Subsec: Implicit Observables}, the geographical location of Sensor Nodes in practice can be reveal by RSSI. Practically speaking, combing this information with the effective radius of RF, the adversary should already be able reconstruct the topology of the network.

Even without the knowledge of the geographical information of Sensor Nodes, we suggest that the topology of the 6LoWPAN network can still be reconstructed by looking at the captured packets:

\begin{description}[style=nextline]
	%With DTLS
	\item With DTLS, the RPL messages are not encrypted. The adversary can reconstruct the network from the captured RPL messages.
	%With noncoresec
	\item With noncoresec, the RPL messages are encrypted. However, the MAC Layer address information is visible to the adversary; therefore the connectivity of Sensor Nodes can still be deduced by looking at the MAC addresses.
\end{description}

\section{Summary}

In this chapter, we started by specifying the objects of leakage we study. There are three objects we study in this report, namely content and size of application data, cryptographic key and network topology.

We categorise the visible data to the adversary in our settings into implicit observables and explicit observables. Implicit observables are meta data of packets including size and timing. Explicit observables are those contents in the unencrypted headers. We also defined the trace in our analysis which contains all packets within a Session. A Session is modelled as an optional Request and a Response.

Finally we give a theoretical analysis to the observables. We argue that if an observable is independent to the secret, then there is no information leakage. Further more, if an observable in the setting is constant, then neither there is any information leakage. We have done the theoretical analysis in two approaches. The first approach is cross referencing the observables exploited in Traffic Analysis Attacks with those applicable in our WSN applications. In the second approach, we semantically analysed the observables based on their definitions in the 6LoWPAN protocol suite and implementation in our platform. 

In conclusion, the potential leakage sources in the observables include:
\begin{enumerate}
	\item The packet features derived from packet size and timing.
	\item ACK Request flag.
	\item Sequence Numbers in different headers are theoretically potential leakage sources, but we suspect they are independent to upper layer applications.
\end{enumerate}

With respect to network topology, the adversary has its full knowledge if the WSN is protected by DTLS. In case of noncoresec, the adversary can still reconstruct the topology at some level based the unencrypted MAC address information. 