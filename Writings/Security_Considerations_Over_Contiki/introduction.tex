\chapter{Introduction}
%Contiki OS is an embedded system that used to build WSN\footnote{Wireless Sensor Network} based on 802.15.4\cite{802154} compatible devices and 6LowPAN\cite{rfc4944}. This paper discusses two security measurements, namely Link Layer Security (LLSEC) and Datagram TLS (DTLS), within Contiki OS. We also discuss some potential methods of fingerprinting an application running on a sensor node.
%
%In \Cref{Chp: LLSEC} we describes a LLSEC implementation in Contiki OS called \textit{noncoresec} and argues that it does not met certain cryptographic security notions.
%
%\Cref{Chp: DTLS} discusses some implementation issues of DTLS on Contiki OS.
%
%\Cref{Chp: Appdetect} first argues that under some natures of WSN, an adversary could possibly collect more accurate timing information than usual Internet Web-application attacker. Then we describe a potential side-channel attack that fingerprints an application the target sensor node is running, using interactions between PING protocol and the application running on the target node.
Recent technology advance has drawn both industrial and academical attention  to the Internet of Things, IoT. Comparing to traditional embedded devices, the latest hardware are more compacted while possessing incomparable computational power. These advantages enabled the devices to perform more computation at the field and thus inspires us to built more intelligent and robust applications, such as smart houses and smart cities. 

Wireless Sensor Network, WSN, is an important building block to many IoT applications as it serves as the raw data input. The recent devices, such as CC2538\cite{CC2538} and Telos B\cite{TelosB}, can not only be used on sensor readings but can also be used as actuators due to their improved computational power. 

Bringing more smart devices into life implies more data being generated. This poses a strong requirement from the aspect of security, as such data can be critical and sensitive. For example, malicious data injected into a transportation system may cause catastrophic results; information breach in a hospital environment can be a severe privacy violation.

Despite the urgent desire of security, it turns out that implementing security measures is a difficult task on WSN devices. The first and greatest challenge is the contradiction between strong security requirements and constrained resources on the devices. This includes bandwidth, computational power, storage and energy, etc, which are less concerned in the existing measurements. The second challenge is the scaled and exposed deployment of these devices. For instance, in a smart city application, the sensors will be deployed all over the city and therefore it would be difficult to prevent an adversary from gaining physical access to the devices, or to passively eavesdrops all its traffic. Above this, the inequality of resources between the device and an adversary makes the problem even more difficult. For example, it does not cost much effort for an adversary who tries to use her desktop to break a sensor in her neighbour’s smart house, which only has an 8-bit processor lives on a battery.

\section{Motivation}
In this project, we are specifically interested in those information that could leak through wireless traffic of WSNs, i.e. we are more interested on those attacks that do not require, or requires only minimum, physical contacts to the victim devices. The motivation is simple and straight forward, as such attacks are usually most practical and threatened in real world scenarios.

\section{Structure}
We will start by introducing some potential options of building blocks for a WSN in \Cref{Chp: LiteratureReview}, followed by a review with regard to some existing remote attacks on Internet that could potentially be referred in the case of WSN, and finally concludes the chapter with packet features that we consider can be potential sources of information leakage. In \Cref{Chp: Setup} we will give a description about our Contiki\cite{Contiki} based experimental environment including the notes of issues that we encountered during the process of setting up such an environment. Then we will focus our analysis on two specific major security measurements that have been implemented on our platform, namely LLSEC in \Cref{Chp: LLSEC} and DTLS in \Cref{Chp: DTLS}. Finally we concludes the report in \Cref{Chp: Conclusion}.