\chapter{Introduction}

Recent advances in embedded device technologies has greatly drawn both industrial and academic attention to Internet of Things, IoT. Comparing to traditional embedded devices, the latest hardware are more compacted in size while possessing incomparable computational power. These improvements have enabled the devices to perform more complicated tasks at field; thus inspiring us to integrate them into things we have never done before and creating more intelligent and robust applications, such as smart houses and smart cities. 

Wireless Sensor Network, WSN, is an important building block to many IoT applications as it serves the role of raw data input. The recent WSN oriented devices, such as CC2538\cite{CC2538} and Telos B\cite{TelosB}, not can only be used on sensor readings but can also be used as actuators due to their improved performance. Meanwhile, several embedded systems with different features are being developed as well to support these hardware, such as Contiki, OpenWSN and RTOS, etc. These together provided a platform allowing developers to deploy their application efficiently, bringing the era of IoT closer to us.
 
Having more smart devices in life implies more data being generated. This poses a strong requirement with respect to security, as these data are sometimes critical and sensitive. For example, malicious data injected into a transportation system may cause catastrophic accidents; information breach in a hospital environment can be a severe privacy violation.

Despite the urgent desire of security in IoT, it turns out that implementing of security measures to be a difficult task on WSN devices. The greatest challenge is the contradictory fact of strong security demand and constrained resources including bandwidth, computational power, storage and energy, etc. These factors are usually less concerned in existing security measures. The second challenge is the scaled and exposed deployment of these devices. Take smart city for instance, the sensors are deployed all over the city and therefore it would be difficult to prevent adversaries from gaining physical access to the devices, or to prevent them from passively eavesdropping the wireless traffic. Moreover, the inequality of resources between the device and adversary makes the problem even more difficult. For example, it does not cost too much effort for an adversary to use a desktop to break a sensor in his neighbour’s smart house, which perhaps only has an 8-bit processor and lives on a small battery.

\section{Motivation}
In this project, we are specifically interested into investigating the information that could leak through wireless traffic in a WSN. In other word, we aimed to study attacks that do not require at all, or only minimum, physical contact to target device. The motivation is simple and straight forward, as such attacks are usually practical and threatening in real world.

\section{Structure}
We start by introducing some building blocks for WSNs in \Cref{Chp: Building Blocks}, followed by a review existing security flaws that could possibly be referred to WSNs \Cref{Chp: LiteratureReview}. Then we cross reference the observable metadata in WSN with those traffic features that have been exploited on Internet to discuss the potential of information leakage sources. In \Cref{Chp: Setup} we give a description about our Contiki\cite{Contiki} based experiment environment including notes of issues we encountered during the setup process.Then we focus on data collected for two major security measures that have been implemented on our platform, namely LLSEC in \Cref{Chp: LLSEC} and DTLS in \Cref{Chp: DTLS}, respectively. In \Cref{Chp: PINGLOAD}, we present an on going study that exploits PING packets to fingerprint the code routine on a sensor node device and its potential usage of deducing some secret values on the target. The concept of this attack derived from a combination of Traffic Analysis techniques and AES cache timing attacks.