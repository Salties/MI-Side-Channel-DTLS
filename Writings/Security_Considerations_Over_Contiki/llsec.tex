\chapter{Link Layer Security: noncoresec} \label{Chp: LLSEC}

In this chapter, we provide an analysis for the Link Layer security measure noncoresec, which we have explained in \Cref{Subsec: 802.15.4 Security Implementation in Contiki}.

\section{Protocol and Implementation}
%Reset Problem
In this section, we analysis the noncoresec implementation with respect to the problems described by \cite{802154sec}. 

\subsection{Nonce Reuse}

As we have explained in \Cref{Subsec: 802154 Nonce}, the only variable field in the nonce is Frame Counter. Since noncoresec only supports network shared key, there are two potential problems according to \cite{802154sec}, namely nonce reuse and anti replay. Further inspecting the source code, we realised that the counter is declared as a static value and is initialised to $0$ on each reboot.

Our experiments confirmed the vulnerability. We simulated two executions of  broadcast\_example of keyllsec in \Cref{Sec: Applications}, broadcasting the same message. Our data\cite{NonceReuseData} showed that frames with the same Frame Counter results into the same ciphertext. In reality, this vulnerability implies that an adversary capable to reset the device can eventually learn the difference in plaintext by calculating the difference of ciphertext with the same Frame Counter, causing severe breach of data confidentiality.

One solution might be to store part of the Frame Counter on the flash and increases that value on each time reboot. Assume the device averagely sends one frame every minute stores the highest byte of Frame Counter, it is resilient in $2^8$ reboots and the lower bytes still has the space of $2^{24}$ frames, which holds up to nearly 32 years. Another solution could be to set the higher bytes of Frame Counter to a random value on each reboot. For example, by setting the highest byte to a uniformly distributed random value in $[0,255]$, the adversary is expected to successfully reset the Frame Counter to a specific value with probability of $2^{-8}$ on each reboot.

\subsection{Anti Replay}

\cite{802154sec} has pointed out that anti replay in 802.15.4 Security is incompatible with network key as the same ACL entry is shared among multiple nodes, causing confusion of Replay Counter in ACL. However, with an inspection of the source code of Contiki, we realised that the noncoresec does not suffer the same problem as the ACL is not implemented; instead, a similar data structure is added to the routing table in kernel  associated to each source address.

\section{Common Analysis}

%Packet Length

%Timing for encryption

%RPL Messages

%Sequence Numbers

\section{Application Analysis}




%
%Link Layer Security, or LLSEC, is a security measure that implements cryptography at Data Link Layer\footnote{https://en.wikipedia.org/wiki/OSI\_model} which is only above Physical Layer.
%
%Introducing cryptography at a lower level has several benefits. Firstly, more data being encrypted reduces the observable packet features to an adversary, such as SRC\footnote{Source Address} and DST\footnote{Destination Address} field in the IP header which are very likely to be exploited by an adversary. Secondly, authentication at lower level also prevents an active adversary from joining the network which therefore weakens her power. 
%
%On the other hand, imposing cryptography at a lower level also brings more challenge to the design of sensor network architecture. The first problem is its overhead. For example, even for a node that only forwards a packet to its next hop, it must decrypt the whole packet to extract its routing information, and then re-encrypt it before transmission. This is particularly problematic in a mesh wireless sensor network as it could potentially downgrades the performance and causing energy consumption problems. More over, key management is also challenging due to the constrained computational power and power optimised lossy nature of wireless sensor network.
%
%It is noticeable that some packet features are not hidden even with LLSEC enabled, such as packet length, timing information and part of the MAC header in a 802.15.4 packet.
%
%\section{802.15.4 Security: {\it noncoresec}} \label{sec: noncoresec}
%{\it noncoresec}\cite{LLSEC} is the current implementation of LLSEC in Contiki. It implements AES\_CCM\_16 ciphersuite in 802.15.4 standard. This section briefly describes how it works.
%
%\begin{itemize}
%\item {\bf Key Management}: All nodes share a network wide AES key for both encryption and authentication. The key is hardcoded during the setup stage.
%
%\item{\bf AEAD\footnote{Authenticated Encryption with Associated Data}}: {\it noncoresec} implements AES\_CCM\_16 \footnote{CCM mode of AES-128 with 16 bytes MAC} as described in 802.15.4\cite{802154} which turns AES into a stream cipher. The same key is used for both encryption and authentication.
%
%\item{\bf Initial Vector (IV, or nonce)}: The IV for each packet is constructed from certain fields of unencrypted MAC header and therefore is public.
%\end{itemize}
%
%An adversary without the knowledge cannot join the sensor network protected by \textit{noncoresec} as she cannot sent out a valid RPL message.
%
%\section{Weak IV}
%
%\begin{figure}
%\centering
%\begin{tabular}{| l | l | l | l | l |}
%\hline
%Flags(1) & Addresses(8) & Frame Counter(4) & Security Level(1) & Block Counter(2)       \\ \hline
%\end{tabular}
%\caption{IV of 802.15.4 Frame with Security} \label{Tbl: 802154 Frame}
%\end{figure}
%
%One problem within the {\it noncoresec} implementation is the low variance of IV. The IV is a $16$ byte bit-string constitutes of the following fields(\Cref{Tbl: 802154 Frame}):
%\begin{itemize}
%\item {\bf Flags (1 byte)}: This field contains part of the MAC header. It is identical to most (basically all) of the data packets.
%
%\item{\bf Source Address (8 bytes)}: This is mapped from the source address field of the packet.
%
%\item{\bf Frame Counter (4 bytes)}: This field increases by 1 from 0 for each packet sent to prevent replay attack.
%
%\item{\bf Security Level (1 byte)}: This field indicates which ciphersuite to be used for this packet. In the case of AES\_CCM\_16, this is constantly 0x7.
%
%\item{\bf Block Counter (2 bytes)}: This field begins from 0x0 and increases by 0x1 for each block in CCM mode. The block length for AES-128 is 16 bytes. The 2 bytes counter is usually sufficient as it supports up to $2^{32}$ bytes of data whereas the minimum MTU\footnote{Maximum Transmit Unit, simply speaking this is the maximum length of a packet.} required by 6lowPAN standard\cite{rfc4944} is $127$ bytes.
%\end{itemize}
%
%In the current {\it noncresec} implementation, \textbf{Flags} and \textbf{Security Level} are constant. \textbf{Block Counter} always begins from 0x0 and the \textbf{Source Address} is also constant for a specific device. Such design leaves the 4 bytes \textbf{Frame Counter} the only field that is variable. This indicates that only $2^{32}$ messages are allowed without a collision of IV which is cryptographically considered to be inappropriate.
%
%\subsection{Reset Problem}
%The low variance of IV leads to a plaintext leakage problem which only requires the adversary to reboot the target node. 
%
%The idea is that rebooting the device resets the \textbf{Frame Counter} to 0x0; hence once a pair of packets with same \textbf{Frame Counter} is found, the difference of their plaintext can be computed by their ciphertext:
%\begin{equation*}
%\Delta p = c_1 \oplus c_2
%\end{equation*}
%where $\Delta p$ is the difference of plaintexts. $c_1$ and $c_2$ are their ciphertext respectively.
%
%\begin{example}
%\begin{figure*}
%\centering
%{
%	\includegraphics[width=0.9\textwidth,]{fig/resetproblem.png} 
%}
%\caption{Captured packets with {\it noncoresec} enabled} \label{Fig: reset problem}
%\end{figure*}
%
%\Cref{Fig: reset problem} demonstrates some packet captured\footnote{The duplicated packets are caused by the retransmission of ContikiMAC\cite{ContikiMAC}.} with {\it noncoresec} enabled. These packets are captured with a sensor broadcasting a 4 byte integer with left side of \Cref{Fig: reset problem} being $[00000000]_{16}$ and right $[12345678]_{16}$. Marked are the corresponding ciphertexts which are $[00127401]_{16}$ and $[12262279]_{16}$ respectively.
%
%As we can see, the difference of ciphertext is exactly the difference of plaintext:
%\begin{equation}
%\Delta p = [00127401]_{16} \oplus [12262269]_{16} = [12345678]_{16}
%\end{equation}
%\end{example}
%
%\section{Distinctive Packet Length for RPL Packets}
%Some RPL packets are shorter than the minimum length of data packets which can be used to distinguish the packets. Further more, some RPL packets set MAC header flags differently from data packets.
%
%\section{Performance Issue}
%The header overhead with LLSEC enabled is 20 bytes which is relatively a large overhead comparing to the 127 bytes MTU requirement of 6LowPAN standard\cite{rfc4944}.