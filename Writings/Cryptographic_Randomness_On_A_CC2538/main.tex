\documentclass{article}

%\usepackage[latin9]{inputenc}
\usepackage{amsmath}
\usepackage{graphicx}
\usepackage{algorithmic}
\usepackage{stfloats}
\usepackage{hyperref}

%Common packages.
\usepackage{cite}
\usepackage{amsmath}
\usepackage{amssymb}
\usepackage{graphicx}
\usepackage{listings}
\usepackage{url}
\usepackage{appendix}
\usepackage{adjustbox}
\usepackage{subcaption}

%NOT TO BE INCLUDED IN SUBMISSION.
\usepackage{hyperref}

%Need to be at last.
\usepackage{cleveref}


\PassOptionsToPackage{hyphens}{url}\usepackage{hyperref}

\title{On the Insecurity of the Randomness Generation in a Large Class of Commercial IoT Devices.}

\author{ 
Yan Yan \\ 
%Department of Computer Science, \\ 
%University of Bristol, UK \\ 
y.yan@bristol.ac.uk
\and
Elisabeth Oswald \\ 
%Department of Computer Science, \\ 
%University of Bristol, UK \\ 
Elisabeth.Oswald@bristol.ac.uk
\and
Theo Tryfonas \\ 
%Department of Civil Engineering, \\ 
%University of Bristol, UK \\
Theo.Tryfonas@bristol.ac.uk
}

\date{University of Bristol, \\ UK \\ \today}

\begin{document}

\maketitle

%\tableofcontents
\begin{abstract}
Smart metering, smart parking, health, environment monitoring, and other applications drive the deployment of the so-called Internet of Things (IoT). Whilst cost and energy efficiency are the main factors that contribute to the popularity of commercial devices in the IoT domain, security features are increasingly desired. Security features typically guarantee authenticity of devices and/or data, as well as confidentiality of data in transit. Our study finds that whilst cryptographic algorithms for confidentiality and authenticity are supported in hardware on a popular class of devices, there is no adequate support for random number generation available. We show how to passively manipulate the on-board source for randomness, and thereby we can completely undermine the security provided by (otherwise) strong cryptographic algorithms, with devastating results. 
\end{abstract}

\section{Introduction}
Applications for the Internet of Things (IoT) flourish, leaving a great desire for not only energy efficient, cheap devices, but also for devices that support basic cryptographic functionality such as confidentiality and/or authenticity. Popular algorithms for confidentiality are e.g. the Advanced Encryption Standard\cite{AES}, and the Elliptic Curve Digital Signature Algorithm\cite{ECDSA}, which, when used in conjunction, enable to establish secure end-to-end channels via e.g. Datagram TLS\cite{DTLS}. 

However whilst AES is a secure block cipher, one might require randomness to turn it into a secure encryption scheme for arbitrary length messages. Somewhat similarly, ECDSA relies on a known-to-be-secure mathematical problem. However, it also requires large and securely generated random numbers. Consequently, when supporting cryptography the secure generation of random numbers is crucial. 

%The prospering IoT applications constantly proposes higher requirements to security where cryptography plays an important role. Among all prerequisite of implementing cryptographic primitives on these IoT devices, a reliable Random Number Generator (RNG) is critical as it is required in most cryptographic algorithms. In practice, a RNG typically involves a Pseudo Random Number Generator (PRNG) seeded by a high entropy physical source.

In 2013, Texas Instruments (TI) launched a new System-on-chip (SoC), the CC2538\cite{CC2538}, featuring secure channels via 802.15.4\cite{802154Standard} support via multiple cryptographic hardware accelerators. Partially because this cryptographic accelerators, projects such as Contiki\cite{Contiki} and OpenWSN\cite{OpenWSN} began to support the CC2538 with enthusiasm. As of writing this paper, the chip features in the suggested list for Zigbee and 6LoWPAN solution on TI's website\cite{ZigbeeProducts}\cite{6LowPANProducts}.

However, despite all the cryptographic hardware support, the CC2538 does not have a RNG dedicated for cryptographic applications; instead, the user manual suggests to use a 16 bit Linear Feedback Shift Register (LFSR) as a PRNG where the seed is generated by the Radio Frequency (RF) module sampling from the radio noise. Whilst the user guide at no point suggests that this method should be used in conjunction with cryptographic algorithms, developers have little choice in the absence of alternatives. Also, in the absence of published attacks, there is often a temptation to ignore warnings such as in \cite{SmartMeterBlog}. 

\subsection{Our Contribution}
We show in this study that this choice proves catastrophic for cryptographic applications, not only because the in-built PRNG has only 16 bit entropy which can be easily predicted, but also because we are able to practically demonstrate how to use radio jamming to bias the seed obtained from the RF module. Consequently, even if the weak in-built PRNG was replaced by a stronger component, the source for the seed could still be tampered with and thus render the system insecure. All the experimental work in this paper was performed on Contiki OS\cite{Contiki} release version 3.0. 
%The related source code can be accessed at \cite{prngtest}.

Our paper is structured as follows. We begin in \Cref{ContikiDriverIssue} with some Contiki RNG driver issues for CC2538. \Cref{LFSR} revises why using a 16 bit LFSR as PRNG is a bad practice and we show how this design flaw can be exploited to break DTLS in \Cref{BreakDTLS}, before reviewing the problem in \Cref{PRNGReflection}. In \Cref{Seed} we explain how CC2538 samples the radio noise into random seed and then we demonstrate how it can be biased by jamming signals in \Cref{Jamming}. Finally we conclude the paper in \Cref{Conclusion}.

\subsection{Related Work}
The design flaw of using a 16 bit LFSR as PRNG has been reported by \cite{SmartMeterBlog}\cite{CC2530PRNG} on CC2430\cite{CC2430Manual} and CC2530\cite{CC2530Manual} respectively. These chips are the predecessors of CC2538 in the SimpleLink\cite{SimpleLink} series and they all adopted the same RNG design. The blogs reported the flaw and warned that it could easily be exploited to break the Z-Stack library\cite{ZStack} and Smart Energy Profile ECC in many Smart Meter applications. We essentially `rediscovered' that this poor design choice still features in the CC2538 product. However, whilst in previous work the possibility of injecting a jamming signal was contemplated, we are the first to actually examine the technical feasibility of this and to demonstrate a working attack.


\subsection{Contiki Driver Issues}\label{ContikiDriverIssue}
We made extensive use of Contiki in our research and fixed (and reported) some coding issues in their RNG driver (release-3.0). These were, the reading out of the LFSR without ready check, a lack of validity check when reading random seed bits from the RF module, and a bug that drops the Most Significant Bit (MSB) and leaves the Least Significant Bit (LSB) to be constantly zero in the seed.  We modified the code and fixed these issues in our experiments. 

Another issue in the driver is that the CC2538 User's Guide\cite{CC2538Manual} suggests only to use the lower byte (8 bits) as a random number but the driver actually used 16 bits in the LFSR. However, this coding mistake does not affect our result, as will be explained in  later sections.
\section{LFSR as PRNG} \label{LFSR}
In this section we explain how CC2538 uses the CRC16 LFSR as a PRNG. 

\begin{figure}[!t]
\centering
\includegraphics[width=2.5in]{fig/crc16.png}
\caption{CRC16 LFSR, from CC2538 User's Guide}
\label{CRC16}
\end{figure}

CC2538 User's Guide describes the PRNG design as: (Section 16.1 in CC2538 User's Guide)
\begin{quote}
The random-number generator is a 16-bit linear-feedback shift register (LFSR) with polynomial X 16 + X 15 +
X 2 + 1 (that is, CRC16). It uses different levels of unrolling depending on the operation it performs. The basic version (no unrolling) is shown in Figure 16-1 (\Cref{CRC16} in this paper).
\end{quote}

When used as a PRNG, the in\_bit of \Cref{CRC16} is constantly $0$. The Contiki driver calls the PRNG by: (Section 16.2.1in CC2538 User's Guide)
\begin{quote}
Another way to update the LFSR is to set the RCTRL bits in the SOC\_ADC\_ADCCON1 register to 01. This clocks the LFSR once (13x unrolling), and the RCTRL bits in the SOC\_ADC\_ADCCON1 register automatically clear when the operation completes.
\end{quote}

In another word, the LFSR is updated by performing 13 CRC16 operations in \Cref{CRC16}.

By nature, the LFSR based PRNG is stateful. Further more, the CRC16 operation is deterministic. Sine there are only 16 bits in the LFSR, we can denote the universal set of its  possible values $\mathbb{S}$ as:
\begin{equation} \label{PRNGState}
\mathbb{S} = \{ S_{i} | S_{i} \in \{2\}^{16}\}
\end{equation}
\Cref{PRNGState} also implies that the LFSR can have no more than $|\mathbb{S}| = 2^{16} = 65536$ values, which is feasible to enumerate.

We also denote the LFSR update operation as:
\begin{equation}
F:\mathbb{S} \rightarrow \mathbb{S}
\end{equation}
where $F$ is $13$ times of the deterministic CRC16 operation.

Denote the random seed sampled by the radio noise as $S^*$. The PRNG can be formalised as:
\begin{equation}
	\begin{aligned}
	S_{0} &= S^* \\
	S_{i+1} &= F(S_{i})
	\end{aligned}
\end{equation}

Since ${S}$ is finite and $F$ is deterministic, the random number sequence will eventually results into a cyclic sequence. The maximum non-repetitive sequence of PRNG output $R$ can be represented as:
\begin{equation}
R= <F^0(S^{*}), F^{1}(S^{*}), ..., F^{n-1}(S^{*})>
\end{equation}
where $S^{*} = F^{0}(S^{*}) = F^{n}(S^{*})$.

The sequence $R$ is determined once the seed $S^{*}$ is given. Since outputs in $R$ are non-repetitive, we have $n \leq |\mathbb{S}| = 65536$.



\section{Broken DTLS}
In this section we explain how to exploit the PRNG flaw to break ECDHE-ECDSA in DTLS handshake protocol.

\section{Reflection}
In this section we propose some idea to implement proper PRNGs on IoT devices.
\section{Seeding by RF Noise} \label{Seed}

Seeding is important in RNG designs as it is fundamental to the randomness of PRNG. CC2538 suggests in its manual to  use the RF core to generate sample seed, as quote as:(Section 16.2.2 in CC2538 User's Guide\cite{CC2538Manual})
\begin{quote}
For the CC2538, when a random value is required, writing the SOC\_ADC\_RNDL register with random bits from the IF\_ADC in the RF receive path seeds the LFSR.
\end{quote}
and: (Section 23.12 in CC2538 User's Guide\cite{CC2538Manual})
\begin{quote}
Single random bits from either the I or Q channel can be read from the RFRND register.
\end{quote}
In case of Contiki, the driver only used the bits generated in I channel.

For the randomness of this seeding method, the manual\cite{CC2538Manual} reported: (Section 23.12 in CC2538 User's Guide\cite{CC2538Manual})
\begin{quote}
Randomness tests show good results for this module. However, a slight DC component exists. In a simple test where the RFRND.IRND register was read a number of times and the data was grouped into bytes, about 20 million bytes were read. When interpreted as unsigned integers between 0 and 255, the mean value was 127.6518, which indicates that there is a DC component.
...
For the first 20 million individual bits, the probability of a 1 is $P(1) = 0.500602$ and $P(0) = 1 - P(1) = 0.499398$.
\end{quote}

Their test results are shown in \Cref{SeedResult}.

\begin{figure}[!t]
\centering
\includegraphics[width=2.5in]{fig/CC2538_Seed1.png}
\includegraphics[width=2.5in]{fig/CC2538_Seed2.png}
\caption{RF core seeding result, from CC2538 User's Guide}
\label{SeedResult}
\end{figure}

To further verify the randomness of this seeding method, we applied the NIST Statistical Test Suite\cite{NISTTest} on 13263600 bits sampled by this seeding method using our application in \cite{prngtest}. Since each read to RFRND generates only $1$ bit, we concatenated all bits into one bit stream of length $13263600$. The bits has passed all tests in the NIST test suite, with $P(0) = 0.49995001$ and $P(1) = 0.50004999$. The full report and raw data are applicable at \cite{prngtest}.

Despite the good randomness of the seed, sampling from RF noise remains sceptical from a security perspective as such physical source can be easily tampered remotely by sending signal wave to the device.

The released documents did not explain further details of how the output of IF\_ADC in the receive I/Q channels translated to random bits. We have neither found any open document describes  the RF design of CC2538. 

However, we noticed the same RNG design has been applied on several product in TI's SimpleLink\texttrademark series. Some of them provided better explanation of their design of RF core and RNG which could be a hint to CC2538. In CC2430 user manual\cite{CC2430Manual}, we found a description of the RF core, shown in \Cref{CC2430RF}.

\begin{figure}[!t]
\centering
\includegraphics[width=2.5in]{fig/CC2430_Radio.png}
\includegraphics[width=2.5in]{fig/CC2430_Demodulator.png}
\caption{CC2430 RF Design, from CC2430 user manual\cite{CC2430Manual}}
\label{CC2430RF}
\end{figure}

\Cref{CC2430RF} suggests that the input analogue signal to IF\_ADC has went through the following components:
\begin{itemize}
	\item Low Noise Amplifier (LNA) which amplifies the signal.
	\item Mixer which down converts the signal frequency. The Frequency Synthesiser is used as the local oscillator.
	\item Band pass filter which filters out the out of band signals.
	\item The Automatic Gain Control (AGC) circuit which further adjusts the signal strength to the input level of ADC.
\end{itemize}

CC2520 Data Sheet\cite{CC2520Manual} explains the random bit is actually the Least Significant Bit (LSB) from ADC: (Section 24 in \cite{CC2520Manual})
\begin{quote}
Single random bits from either the I or Q channel (configurable) can be output on GPIO pins at a rate of 8MHz. One can also select to xor the I and Q bits before they are output on a GPIO pin. These bits are taken from the least significant bit in the I and/or Q channel after the decimation filter in the demodulator.
\end{quote}

A block diagram is also provided, as shown in \Cref{CC2520RFRND}.
\begin{figure}[!t]
\centering
\includegraphics[width=2.5in]{fig/CC2520_RNG.png}
\caption{CC2520 RNG Design, from CC2520 user manual\cite{CC2520Manual}}
\label{CC2520RFRND}
\end{figure}

Interestingly, we noticed that CC2538, CC2520, CC253X and CC2540/41 reported exactly the identical randomness test result in their user manuals (\cite{CC2538Manual} \cite{ CC2520Manual} \cite{CC2530Manual}). This suggests they are very likely to have the same seeding design.

Assuming the same design has been applied to CC2538, it would explained the nice randomness of the seeding method. Denote $V$ as analogue RF signal and $N$ as noise, the analogue input to the ADC $V_{in}$ can be represented as:
\begin{equation}
V_{in} = V + N
\end{equation}

The noise $N$ can be induced by multiple sources in practice, including:
\begin{itemize}
\item Noise produced by the signal source.
\item Environmental noise.
\item Noise induced by the components in the device itself.
\end{itemize}
In practice, manipulating the noise could be difficult.

The random bit $b$ can be represented as:
\begin{equation} \label{RNDOutput}
b = LSB(V_{in}) = LSB(V + N)
\end{equation}
where $LSB() \in \{0,1\}$ represents the operation of taking the LSB of A/D conversion output.

Observing \Cref{RNDOutput}, one thing to be noticed is that any difference in $V_{in}$ larger than the scale of ADC, i.e. the voltage represented by its LSB, could flip $b$. According to CC2538 data sheet\cite{CC2538Datasheet}, the receiver can be sensitive to signals down to $-97dBm$ (typical value with $T_A = 25^{\circ}C$, $V_{DD} = 3V$ and $f_{C} = 2440MHz$). On the other hand, the typical environmental noise in our experimental environment, which is a typical office with multiple noise sources such as  WiFi, smart phones, etc, is about $-92dBm$ which is significantly higher than the receiver sensitivity. We consider the result of the randomness test as an evidence to this sampling method.

\section{Biasing Seed by Radio Jamming}
At a first glance one way to bias the seed is to generate a predictable $V_{in}$. \Cref{RNDOutput} indicates that the random bit $b$ is jointly determined by the signal $V$ and noise $N$. Even though $V$ can be viewed as being controlled by the adversary, manipulating $N$ turns out to be difficult in practice. For instance, noises accumulated by different amplification stages are physically inevitable. Hence fixing $V_{in}$ does not seem to be easily achievable in practice.


\section{Conclusion}
In this paper we presented a study on the RNG design of CC2538. First, we revised the problem that using a 16 bit LFSR as PRNG is a bad idea and demonstrated how this design flaw can be exploited to break DTLS running on these devices. Secondly we presented a study to its seeding method and showed how such it could be remotely tampered by an adversary sending jamming signal to the device.

In fact the same RNG design has also been adopted many other products in the CC series including CC2420\cite{CC2420Manual}, CC2430\cite{CC2430Manual}, CC2520\cite{CC2520Manual} and CC253X, CC2540/41 series\cite{CC2530Manual}. We imagine all these products suffer the same problems. Fortunately the latest CC26XX/CC13XX\cite{CC26XXManual} has abandoned this design and implemented a dedicated RNG which TI describes as: (Chapter 16 in CC26XX/CC13XX Manual\cite{CC26XXManual})
\begin{quote}
The true random number generator (TRNG) module provides a true, nondeterministic noise source for the
purpose of generating keys, initialization vectors (IVs), and other random number requirements. The
TRNG is built on 24 ring oscillators that create unpredictable output to feed a complex nonlinear
combinatorial circuit. That post-processing of the output data is required to obtain cryptographically secure
random data.
\end{quote}

We sincerely hope this TRNG will provide the future IoT applications a secure RNG.

\section{Acknowledgement}
We have many thanks to (alphabetically) George Oikonomou for providing us much help in Contiki OS and the OpenMote devices, Geoff Hilton who helped us on RF designs and Jake Longo Galea who offered many signal processing advises.
\section{Related Source Code}

Source code related to this paper can be found at \cite{cc2538rng}. The repository contains all source file to reproduce the experiments done in this paper, including:
\begin{itemize}
	\item \textbf{prngtest}: Contiki application that iterates the CC2538 PRNG.
	\item \textbf{genr.py} and \textbf{secp256r1mult}: Tools to generate the EC key pair lookup table for CC2538 PRNG.
	\item \textbf{cc2538seed}: Contiki application that samples the RF seed for CC2538.
	\item \textbf{biasseed.py}: Implementation of the strong constant signal for HackRF One.
\end{itemize}

Detail usage documented in readme.txt. %Only included for author's version.

\bibliographystyle{IEEEtran}
\bibliography{references}

\appendix
\chapter{Formal Proof of \Cref{Te: IR}} \label{Prf: IR}
\begin{proof}
	%Independent random variables does not leak.
	Since $X$ and $Y$ are independent, therefore
	\begin{eqnarray*}
		\begin{aligned}
			P(x,y) &= P(x)P(y) \\
			P(x|y) &= P(x)
		\end{aligned}
	\end{eqnarray*}
	where $x \in X$ and $y \in Y$.

	For Mutual Information and Capacity, we have:
	\begin{eqnarray*}
		\begin{aligned}
			H(X|Y) 
			&= - \sum_{x \in X} \sum_{y \in Y} P(x,y)\log{P(x|y)} \\
			&= - \sum_{x \in X} \sum_{y \in Y} P(x)P(y)\log{P(x)} \\
			&= \sum_{y \in Y} P(y) (- \sum_{x \in X}P(x)\log{P(x)}) \\
			&= \sum_{y \in Y} P(y) H(X) = H(X) \sum_{y \in Y}{P(y)} \\
			&= H(X)
		\end{aligned}
	\end{eqnarray*}
	
	Therefore
	\begin{eqnarray*}
		\begin{aligned}
			I(X;Y) &= H(X) - H(X|Y) = H(X) - H(X) = 0 \\
			C &= \sup_{\forall P(X)} I(X;Y) = \sup_{\forall P(X)} 0 = 0
		\end{aligned}
	\end{eqnarray*}
	
	Similarly for gain function based leakage\cite{GLeakage},
	\begin{eqnarray*}
		\begin{aligned}
			V_{g}(\pi, C) 
			&= \sum_{y \in Y}{\max_{w \in W}\sum_{x \in X}{\pi[x]C[x,y]g(w,x)}} \\
			&= \sum_{y \in Y}{\max_{w \in W}\sum_{x \in X}{\pi[x]P(y|x)g(w,x)}} \\
			&= \sum_{y \in Y}{\max_{w \in W}\sum_{x \in X}{\pi[x]P(y)g(w,x)}} \\
			&= \sum_{y \in Y}p(y){\max_{w \in W}\sum_{x \in X}{\pi[x]g(w,x)}} \\
			&= \max_{w \in W}\sum_{x \in X}{\pi[x]g(w,x)} = V_{g}(\pi)
		\end{aligned}
	\end{eqnarray*}
	
	Therefore
	\begin{equation*}
		H_g(\pi, C) = -\log{V_g(\pi, C)} = -\log{V_g(\pi)} = H_g(\pi)
	\end{equation*}
	
	Hence 
	\begin{eqnarray*}
		\begin{aligned}
			L_g(\pi, C) &= H_g(\pi) - H_g(\pi,C) = H_g(\pi) - H_g(\pi) = 0\\
			ML_g(C) &= \sup_{\pi} L_g(\pi, C) = \sup_{\pi} 0 = 0
		\end{aligned}
	\end{eqnarray*}
\end{proof}



\chapter{Details of Packet Feature Cross Reference} \label{Detail Cross Reference}

For the exploited traffic features in 

\begin{description}[style=nextline]
	\item[Direction]
	In our applications, the directions of packet is a predictable constant. We consider this is not a 
	
	\item[Length]
	The is effectively the packet size in implicit observables.
	
	\item[Frequency Distribution of Length]
	The same feature can be computed by packet sizes. However, since there are typically only two packets in a trace, the result is $0.5$ for the length of Request packet and $0.5$ for the length of Response packet. In a one packet Session there is only one value in the distribution with probability of $1$. This feature is applicable but with extremely low entropy of $1$ or $0$.
	
	\item[Size, HTML and Number Markers]
	In a two packet Session there is only one direction change in a trace; thus the markers constantly mark the second packet. In an one packet Session this feature is not applicable.
	
	\item[Total Bytes]
	The same feature can be computed through packet sizes.
	
	\item[Percentage Incoming Packets]
	The term ``incoming'' refers to the direction of web server to the browser in its original Web Fingerprint literature. In our experiments we assumed the adversary monitors all packets in the network; thus there is not an explicit definition of ``incoming'' and ``outgoing''. Even though we can similarly define ``incoming'' as from Sensor Node to Manager, this is feature is fixed given an application. This value is constantly $50\%$ for a two packets Session and $100\%$ for a one packet Session.
	
	\item[Number of Packets]
	Since UDP does not segment any application data, the number of packets in a trace is a constant given an application. 
	
	\item[Total Time]
	In a two packets Session this is exactly the interval between Request and Response. In an one packet session this is not applicable.
	
	\item[Total Per-direction Bandwidth]
	Since there is at most only one packet at each direction, this feature is effectively a single packet size divided by total time for each direction.
	
	\item[Traffic Burst]
	Traffic burst is reduced to packet size in our applications as there is at most only one packet each direction.
\end{description}

Notice that we ignored Traffic Bursts since it is reduced to packet length in our applications as explained above.

According to \Cref{Cor: Constant Leakage}, features with constant value are non leakable features. 

\chapter{Leakage of Linear Packet Size} \label{Linear Leakage}

Modelling the leakage of packet length as a channel $C(l_{C},l_{P})$ as in other Information Theoretic approaches we described in \Cref{Subsec: Information Theory}, we have a deterministic channel such that:

\begin{equation}
	C(l_{P}, l_{C}) = P(l_{P} | l_{C}) = 
	\begin{cases}
		1 &\text{if: } l_{C} = l_{P} + b \\
		0 &\text{otherwise}
	\end{cases}
\end{equation}

and

\begin{equation}
	C^{-1}(l_{C}, l_{P}) = P(l_{C} | l_{P}) = 
	\begin{cases}
		1 &\text{if: } l_{C} = l_{P} + b \\
		0 &\text{otherwise}
	\end{cases}
\end{equation}

So

\begin{equation}
	P(l_{P} , l_{C}) = P(l_{P}) P(l_{C} | l_{P}) =
	\begin{cases}
		P(l_{P}) &\text{if: } l_{C} = l_{P} + b \\
		0 &\text{otherwise}
	\end{cases}
\end{equation}

Therefore\footnote{Information Theory defines $0\log{0} = 0$.},
\begin{equation}
	P(l_{P} , l_{C}) \log{P(l_{P} | l_{C})} = 
	\begin{cases}
		P(l_{P})\log{1} = 0 &\text{if: } l_{C} = l_{P} + b \\
		0 \log{0} = 0 &\text{otherwise}
	\end{cases}
\end{equation}

Hence
\begin{equation}
	H(L_{P} | L_{C}) = - \sum_{l_{C} \in L_{C}} \sum_{l_{P} \in L_{P}}P(l_{P} , l_{C}) \log{P(l_{P} | l_{C})} = - \sum_{l_{C} \in L_{C}} \sum_{l_{P} \in L_{P}} 0 = 0
\end{equation}
where $L_{P}$ and $L_{C}$ are the possible length in bytes of encrypted and unencrypted packets.

In this case, the Mutual Information is:
\begin{equation} \label{Eq: MI in length}
	I(L_{P};L_{C}) = H(L_{P}) - H(L_{P} | L_{C} ) = H(L_{P}) - 0 = H(L_{P})
\end{equation}

For the Capacity, according to \Cref{Eq: MI in length}, $I(L_{P};L_{C})$ has its maximum value when $L_{P}$ is uniformly distributed:
\begin{equation} \label{Eq: Cap in length}
	Capacity = \sup_{\forall L_{P}}{I(L_{P};L_{C})} = \sup_{\forall L_{P}}H(L_{P}) = - \sum_{i = 1}^{|L_{P}|}|L_{P}|^{-1}\log{|L_{P}|^{-1}} = \log{|L_{P}|}
\end{equation}

In another word, \Cref{Eq: MI in length} and \Cref{Eq: Cap in length}  imply that averagely all bits of $l_{P}$ is leaked through $l_{C}$.

For the gain function based leakage\cite{GLeakage}, we realised that it would be hard to quantify the leakage without a specific gain function. Therefore instead, we provide an analysis with min-leakage.

In this case, the Posterior Vulnerability is:
\begin{equation}
	\begin{aligned}
		V(\pi_{L_P}, C^{-1}) 
		&= \sum_{l_{C} \in L_{C}} \max_{l_{P} \in L_{P}} \pi_{L_P}[l_P]C^{-1}[l_P,l_C] \\
		&=  \sum_{l_{C} \in L_{C}} P(l_C) \max_{l_{P} \in L_{P}} P(l_P | l_C) \\
	      &= \sum_{l_{C} \in L_{C}} P(l_C) = 1 \\
	\end{aligned}
\end{equation}

Therefore
\begin{equation}
	\begin{aligned}
		H_{\infty}(\pi_{L_{P}}, C^{-1})
		 &= - \log{V(\pi_{L_{P}}, C^{-1})} = - \log1= 0
	\end{aligned}
\end{equation}

Thus the min-leakage is:
\begin{equation}
	\begin{aligned}
	L(\pi_{L_P}, C^{-1}) 
	 &= H_{\infty}(\pi_{L_P}) - H_{\infty}(\pi_{L_{P}}, C^{-1}) \\
	 &= H_{\infty}(\pi_{L_P}) - 0 \\
	 &= H_{\infty}(\pi_{L_P})
	\end{aligned}
\end{equation}

And finally:
\begin{equation}
	ML(C^{-1}) = \sup_{\pi_{L_P}}{L(\pi_{L_P},C^{-1})} =  \sup_{\pi_{L_P}} H_{\infty}(\pi_{L_P}) = \log{|L_P|}
\end{equation}

This result consists with our intuition and the Capacity in \Cref{Eq: Cap in length} that all bits of $l_P$ are leaked through $l_C$.

\end{document}
