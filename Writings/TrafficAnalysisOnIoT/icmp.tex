\section{Extract ICMP Messages}
%Introduction
ICMP messages in 6LoWPAN (ICMPv6) are defined by \cite{rfc4443}. These packets serve the purpose of network maintenance, such as exchanging routing information or network debugging. Although revealing these messages does not directly breach application data confidentially, in IoT applications, such information may still provide adversaries advantages, for example knowing the position of root node might aggravate the damage of Denial of Services (DoS) attacks. Unfortunately, the latest Contiki does not have authentication and encryption implemented for ICMP messages as specified by \cite{rfc2463}, leaving noncoresec the only option for their protection. But even with noncoresec, some packet meta data are still accessible in the sniffed data, including packet size and unencrypted 802.15.4 MAC header.

We simulated a 6LoWPAN application constituted of multiple Wismote\cite{Wismote} nodes running Contiki broadcast and unicast examples. The ICMP messages generated in our simulation includes:
\begin{itemize}
	\item \textbf{DAG Information Object (DIO)} \\
	DIO contains the 6LoWPAN global information. It could be periodically broadcasted for network maintenance, or unicasted to a new joining node as a reply to DIS (see below).
	\item \textbf{DAG Information Solicitation (DIS)} \\
	DIS is sent by a new started node probing any existing 6LoWPANs and requesting their global information to join in. A DIO is replied if it is received by any neighbour nodes.
	\item \textbf{Destination Advertisement Object (DAO)} \\
	DAO is sent by a child node to its precedents to propagate its routing information\footnote{The 6LoWPAN DODAG topology is defined in \cite{rfc6550}.}. This information is later used when routing packets to the child.
	\item \textbf{Neighbour Solicitation (NS) and Neighbour Advertisement (NA)} \\
	NS is sent upon a node querying the associated MAC address to an IPv6 address and NA is replied as the answer to NS. In addition, these messages are also used for local link validity check.
	\item \textbf{Echo Request and Echo Response (PING)} \\
	Echo Request and Echo Response are also well known as the PING packets. They are mostly used for diagnostic purposes, such as connectivity test or Round Trip Time (RTT) estimation. Echo Request may contain arbitrary user defined data and Echo Response simply echoes its corresponding request.
\end{itemize}

DIO/DIS/DAO and NS/NA are defined by \cite{rfc6550} and \cite{rfc6775} respectively. PING is defined by \cite{rfc2463}.

Our simulation shows that even though the exact content of these messages are hidden under the encryption provided by noncoresec, some of them are still distinguishable simply by packet size and type of MAC destination, as summarised in \Cref{ICMPPacketFeature} where $x \in \mathbb{Z}_{0}^{+}$ is the size of user defined data in PING packets:

\begin{itemize}
	\item Both DIO and NS can be sent in either broadcast or unicast. The broadcasted DIO is smaller than unicasted as it uses an abbreviated IPv6 multicast address ``ff02::1a''. Broadcasted NS uses another multicast address ``ff02::1:ff00:0'' which has the same length as an unicast address. However, both of them are mapped to the same ``0xffff'' Link Layer broadcast address in 802.15.4 MAC Header.
	%ICMP ECHO fragmentation in Wismote network.
	\item The size of PING may vary due to different user defined data. According to \cite{rfc4944}, any packet less than the 802.15.4 MTU, i.e. 127 bytes, should not be fragmented; however, we realised that Contiki fragments PING larger than $107$ bytes. We have not identified the cause but we consider this might be an implementation bug.
	%No NS and NA in Sky network.
	%\item Different ICMPv6 messages maybe observed on different platforms. In our experiments, there is no NS and therefore NA observed in WSN built with TelosB.
\end{itemize}

\begin{table}[ht!]
	\center
	\adjustbox{max width = \textwidth}
	{
		\begin{tabular}{|c|c|c|}
			\hline
			       & Packet Size (bytes) & Type of MAC Destination \\ \hline
			DIS    & 85                  & broadcast                       \\ \hline
			DIO  & 118/123                 & broadcast/unicast                       \\ \hline
			DAO    & 97                  & unicast                      \\ \hline
			NS & 87                  & broadcast/unicast                       \\ \hline
			NA     & 87                  & unicast                      \\ \hline
			PING   & $101+x$               & unicast                      \\ \hline
		\end{tabular}
	}
	\caption{ICMPv6 Packets in Simulated 6LoWPAN with noncoresec}
	\label{ICMPPacketFeature}
\end{table}

%UDP packet size.
In practice, the captured traffic contains noises from upper layer applications. In case of the generally used UDP\cite{rfc768}, the packet features are summarised in \Cref{UDPPacketFeature}.

\begin{table}[ht!]
	\center
	\adjustbox{max width = \textwidth}
	{
		\begin{tabular}{|c|c|c|}
			\hline
			       & Packet Size (bytes) & Type of MAC Destination \\ \hline
			UDP Multicast   & $85+x$                  & broadcast                       \\ \hline
			UDP Unicast   & $107+x$                  & unicast                       \\ \hline
		\end{tabular}
	}
	\caption{UDP Packets in Simulated 6LoWPAN with noncoresec}
	\label{UDPPacketFeature}
\end{table}

Comparing \Cref{ICMPPacketFeature} to \Cref{UDPPacketFeature}, the definitely identifiable ICMP packets are summarised in \Cref{TAICMP}.

\begin{table}[ht!]
	\center
	\adjustbox{max width = \textwidth}
	{
		\begin{tabular}{|c|c|c|}
			\hline
			ICMP Packet & (Size, MAC Destination)\\ \hline
			DIS   & (85, broadcast)                       \\ \hline
			DAO   & (97, unicast)                       \\ \hline
			NS (broadcast)   & (87, broadcast)                       \\ \hline	
			NS (unicast)   & (87, unicast)                       \\ \hline
			NA 	& (87, unicast)                       \\ \hline
		\end{tabular}
	}
	\caption{Identifiable ICMP Packets}
	\label{TAICMP}
\end{table}

Other packets in \Cref{ICMPPacketFeature} may still be identifiable in the encrypted traffic, depending on the application. One thing to be noticed is that packet fragmentation may induce false positives into \Cref{TAICMP}, although in practice it is avoided by many implementations.

\subsection{Tracking Root Node}
\textbf{TO BE DONE...}
