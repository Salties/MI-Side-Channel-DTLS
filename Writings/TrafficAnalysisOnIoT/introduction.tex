\section{Introduction}
Traffic Analysis (TA) is a well studied technique that breaches data confidentiality over Internet, typically targeting protocols including HTTPS\cite{rfc2818} and SSL\cite{rfc6101}/TLS\cite{rfc5246}. This type of attacks works by correlating the encrypted contents to side channel information frequently omitted by cryptographic scheme designs, such as packet length, timing and any other unprotected meta data. Consequently, TA is often associated to Internet privacy violation and Mass Surveillance.

The extensive use of radio communication in exposed environment poses a great security challenge to IoT applications, especially against TA attacks as packets containing critical privacy data, e.g. driving information in Vehicular Ad Hoc Networks(VANETs)\cite{VANET} and daily life data in a smart house, can be easily eavesdropped and analysed by adversaries.

In this paper, we discuss the applicability of TA techniques over IoT application traffic. To be more specifically, we present an attempt of TA attacks on a 6LoWPAN\cite{rfc4944} network powered by Contiki\cite{Contiki} and demonstrate the potential of extracting ``supposedly'' private information from. Although revealing these information does not directly imply any security problem in the setup due to the fact that they are purely designed for experimental purposes, but the result indicates that TA attacks should be concerned when designing secure IoT applications.

%Finally the paper structure
We first review the related literatures in \Cref{RelatedWork} and explain an implementation flaw in the current Contiki source code in \Cref{noncoresec}. We then present an example that extracts ICMP messages in 6LoWPAN in \Cref{ICMPAttack}, followed by another attack that reveals hardware information by timing PING packets and their responses in \Cref{Sec:DistinguishDevice}. In \Cref{PingLoad} we propose a new type of side channel attack, named PingLoad, that fingerprints the application that runs on a node, before we conclude the paper in \Cref{conclusion}.

\section{Related Work \label{RelatedWork}}
%Literatures about Traffic Analysis...
Traffic Analysis is a well studied subject in the Internet security and privacy community. \cite{WebSidechannel} summarised several scenario of side channel attacks against web applications, followed by \cite{PinpointWeb} which proposes the use of Mutual Information to pinpoint the potential points of information leakage. \cite{AppleMsg}, \cite{Language} and \cite{VideoTraffic} described attacks against encrypted text, voice and video traffic respectively, whilst \cite{SuggestBox} presented an attack against Google search box. \cite{HClassifier} and \cite{Peekaboo} studied different classifiers when applying Machine Learning on Traffic Analysis. Different countermeasures are proposed by \cite{TrafficMorphing}, \cite{HTTPOS} and \cite{FTE}, etc.

%Literatures about 6LoWPAN security
Other literatures discuss security issues in 6LoWPAN networks. \cite{802154SecIssues} reports some design flaws in 802.15.4 security. \cite{6LoWPANAtk} summarises known security issues in 6LoWPAN networks, including Fragmentation Attack\cite{FragAtk}, Sinkhole Attack\cite{Sinkhole}, Hello Flood Attack\cite{HelloFlood}, Wormhole Attack\cite{Wormhole} and Blackhole Attack\cite{Blackhole}, etc.