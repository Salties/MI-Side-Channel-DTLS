\section{Noncoresec}
%Noncoresec
Noncoresec\cite{noncoresec} is the implemented Link Layer\cite{OSI} security in the latest Contiki release (3.0). Noncoresec is a 802.15.4 Security\cite{802154} implementation using hardcoded network key. Due to the absence of IPSec\cite{rfc4301}, noncoresec is effectively the only applicable approach for securing the network metadata, including IP headers and TCP/UDP headers.

%Platform
%Most of our studies are done on traffic data generated by Cooja simulator\cite{Contiki}. However, TelosB\cite{TelosB} and CC2538\cite{CC2538} are also used to gather performance critical data, such as timings.

%Reset Attack and Anti-Replay
Examining the code, we realised two issues exist in the noncoresec implementation:
\begin{itemize}
	\item \textbf{Nonce Reuse} \\
	According to the 802.15.4 specification\cite{802154}, for packets sent from a specific MAC address, Frame Counter (shown in \Cref{NoncoresecNonce}) is the only variable field in the nonce construction. Noncoresec implemented Frame Counter as a static variable; therefore it is initialised to 0 upon each reboot. Such implementation would allow an adversary to force nonce reuse by resetting the device which in many cases can lead to full breach of the plaintext.
	
	\begin{figure}[th!]
	\centering
	\adjustbox{max width = \textwidth}
	{
		\begin{tabular}{|c|c|c|c|c|}
			\hline 
			1 (bytes) & 8              & 4             & 1              & 2             \\ \hline
			Flags      & Source Address & Frame Counter & Security Level & Block Counter \\ \hline
		\end{tabular}
	}
	\caption{Noncoresec nonce construction}
	\label{NoncoresecNonce}
	\end{figure}
	
	\item \textbf{Anti-replay} \\
	Even though the incompatibility between Anti-replay and network key has been pointed out by \cite{802154SecIssues}, the noncoresec implementation has mitigated the issue by using an extra data structure recording the last Frame Counter from each source address. However, this approach also induces a problem that packets sent by a rebooted device will be labelled as replays and thus ignored.
\end{itemize}

Nonce reusing could be mitigated by initialise Frame Counter to a random value on each reboot or record its latest value on a non loss media such as flash memories, but the 4 bytes space of Frame Counter can hardly be considered cryptographically secure. Therefore a full solution may not be achieved without updating the nonce construction specified by the standard\cite{802154}. The anti-replay issue is closely related to the key management which remains an open question in many IoT applications; hence it would be difficult derive a general solution. Our recommendation is to simply disable the anti-replay feature in noncoresec and leave it to upper layer protocols and/or applications.